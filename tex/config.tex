\documentclass[10pt,landscape,twocolumn,a4paper,notitlepage]{article}
\usepackage{hyperref}
\usepackage[english, activeacute]{babel}
\usepackage[utf8]{inputenc}
\usepackage{fancyhdr}
\usepackage{lastpage}
\usepackage{listings}
\usepackage{amssymb}
\usepackage[usenames,dvipsnames]{color}
\usepackage{graphicx}
\usepackage{wrapfig}
\usepackage{amsmath}
\usepackage{makeidx}
\usepackage{stmaryrd}
\usepackage{mathtools} % Asegura que genfrac esté disponible

% Números de Stirling de primera y segunda especie
\newcommand{\stirlingI}[2]{\genfrac{[}{]}{0pt}{}{#1}{#2}}    % [n k]
\newcommand{\stirlingII}[2]{\genfrac{\{}{\}}{0pt}{}{#1}{#2}} % {n k}

% Números de Euler (primera especie)
\newcommand{\euler}[2]{\left\langle\!\!\left\langle {#1 \atop #2} \right\rangle\!\!\right\rangle}

\usepackage{amsmath, amssymb} % Si aún no lo tienes

% Valor absoluto o cardinalidad
\newcommand{\abs}[1]{\left|#1\right|}

%\usepackage{minted}
\lstset{
  inputencoding=utf8,
  extendedchars=true
}

%%% Margenes
\setlength{\columnsep}{0.25in}    % default=10pt
\setlength{\columnseprule}{0.5pt}    % default=0pt (no line)

\addtolength{\textheight}{2.35in}
\addtolength{\topmargin}{-0.9in}     % ~ -0.5 del incremento anterior

\addtolength{\textwidth}{1.1in}
\addtolength{\oddsidemargin}{-0.55in} % -0.5 del incremento anterior

\setlength{\headsep}{0.08in}
\setlength{\parskip}{0in}
\setlength{\headheight}{15pt}
\setlength{\parindent}{0mm}

%%% Encabezado y pie de pagina
\pagestyle{fancy}
\fancyhead[LO]{\textbf{\title}}
\fancyhead[C]{\leftmark\ -\ \rightmark}
\fancyhead[RO]{P\'agina \thepage\ de \pageref{LastPage}}
\renewcommand{\headrulewidth}{0.4pt}
\fancyfoot{}
\definecolor{darkblue}{rgb}{0,0,0.4}
%%% Configuracion de Listings
\lstloadlanguages{C++}
\lstnewenvironment{code}
{%\lstset{	numbers=none, frame=lines, basicstyle=\small\ttfamily, }%
	\csname lst@SetFirstLabel\endcsname}
{\csname lst@SaveFirstLabel\endcsname}
\lstset{% general command to set parameter(s)
	language=C++, basicstyle=\small\ttfamily, keywordstyle=\slshape,
	emph=[1]{tipo,usa}, emphstyle={[1]\sffamily\bfseries},
	morekeywords={tint,forn,forsn},
	basewidth={0.47em,0.40em},
	columns=fixed, fontadjust, resetmargins, xrightmargin=5pt, xleftmargin=15pt,
	flexiblecolumns=false, tabsize=2, breaklines,	breakatwhitespace=false, extendedchars=true,
	numbers=left, numberstyle=\tiny, stepnumber=1, numbersep=9pt,
	frame=l, framesep=3pt,
	basicstyle=\ttfamily,
	keywordstyle=\color{darkblue}\ttfamily,
	stringstyle=\color{magenta}\ttfamily,
	commentstyle=\color{RedOrange}\ttfamily,
	morecomment=[l][\color{OliveGreen}]{\#}
}

\lstdefinestyle{C++}{
	language=C++, basicstyle=\small\ttfamily, keywordstyle=\slshape,
	emph=[1]{tipo,usa,tipo2}, emphstyle={[1]\sffamily\bfseries},
	morekeywords={tint,forn,forsn},
	basewidth={0.47em,0.40em},
	columns=fixed, fontadjust, resetmargins, xrightmargin=5pt, xleftmargin=15pt,
	flexiblecolumns=false, tabsize=2, breaklines,	breakatwhitespace=false, extendedchars=true,
	numbers=left, numberstyle=\tiny, stepnumber=1, numbersep=9pt,
	frame=l, framesep=3pt,
	basicstyle=\ttfamily,
	keywordstyle=\color{darkblue}\ttfamily,
	stringstyle=\color{magenta}\ttfamily,
	commentstyle=\color{RedOrange}\ttfamily,
	morecomment=[l][\color{OliveGreen}]{\#}
}


\lstloadlanguages{bash}

% Define a custom style for Bash code
\lstdefinestyle{bash}{
	language=bash, basicstyle=\small\ttfamily, keywordstyle=\slshape,
emph=[1]{tipo,usa,tipo2}, emphstyle={[1]\sffamily\bfseries},
morekeywords={tint,forn,forsn},
basewidth={0.47em,0.40em},
columns=fixed, fontadjust, resetmargins, xrightmargin=5pt, xleftmargin=15pt,
flexiblecolumns=false, tabsize=2, breaklines,	breakatwhitespace=false, extendedchars=true,
numbers=left, numberstyle=\tiny, stepnumber=1, numbersep=9pt,
frame=l, framesep=3pt,
basicstyle=\ttfamily,
keywordstyle=\color{darkblue}\ttfamily,
stringstyle=\color{magenta}\ttfamily,
commentstyle=\color{RedOrange}\ttfamily,
morecomment=[l][\color{OliveGreen}]{\#}
}

%%% Macros
\def\nbtitle#1{\begin{Large}\begin{center}\textbf{#1}\end{center}\end{Large}}
\def\nbsection#1{\section{#1}}
\def\nbsubsection#1{\subsection{#1}}
\def\nbcoment#1{\begin{small}\textbf{#1}\end{small}}
\newcommand{\comb}[2]{\left( \begin{array}{c} #1 \\ #2 \end{array}\right)}
\def\complexity#1{\texorpdfstring{$\mathcal{O}(#1)$}{O(#1)}}
\newcommand\cppfile[2][]{
	\lstinputlisting[style=C++]{#2}
}
\newcommand\bashfile[2][]{
	\lstinputlisting[style=bash]{#2}
}