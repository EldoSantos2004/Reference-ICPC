\documentclass[10pt,landscape,twocolumn,a4paper,notitlepage]{article}
\usepackage{hyperref}
\usepackage[english, activeacute]{babel}
\usepackage[utf8]{inputenc}
\usepackage{fancyhdr}
\usepackage{lastpage}
\usepackage{listings}
\usepackage{amssymb}
\usepackage[usenames,dvipsnames]{color}
\usepackage{graphicx}
\usepackage{wrapfig}
\usepackage{amsmath}
\usepackage{makeidx}
\usepackage{stmaryrd}
\usepackage{mathtools} % Asegura que genfrac esté disponible

% Números de Stirling de primera y segunda especie
\newcommand{\stirlingI}[2]{\genfrac{[}{]}{0pt}{}{#1}{#2}}    % [n k]
\newcommand{\stirlingII}[2]{\genfrac{\{}{\}}{0pt}{}{#1}{#2}} % {n k}

% Números de Euler (primera especie)
\newcommand{\euler}[2]{\left\langle\!\!\left\langle {#1 \atop #2} \right\rangle\!\!\right\rangle}

\usepackage{amsmath, amssymb} % Si aún no lo tienes

% Valor absoluto o cardinalidad
\newcommand{\abs}[1]{\left|#1\right|}

%\usepackage{minted}
\lstset{
  inputencoding=utf8,
  extendedchars=true
}

%%% Margenes
\setlength{\columnsep}{0.25in}    % default=10pt
\setlength{\columnseprule}{0.5pt}    % default=0pt (no line)

\addtolength{\textheight}{2.35in}
\addtolength{\topmargin}{-0.9in}     % ~ -0.5 del incremento anterior

\addtolength{\textwidth}{1.1in}
\addtolength{\oddsidemargin}{-0.55in} % -0.5 del incremento anterior

\setlength{\headsep}{0.08in}
\setlength{\parskip}{0in}
\setlength{\headheight}{15pt}
\setlength{\parindent}{0mm}

%%% Encabezado y pie de pagina
\pagestyle{fancy}
\fancyhead[LO]{\textbf{\title}}
\fancyhead[C]{\leftmark\ -\ \rightmark}
\fancyhead[RO]{P\'agina \thepage\ de \pageref{LastPage}}
\renewcommand{\headrulewidth}{0.4pt}
\fancyfoot{}
\definecolor{darkblue}{rgb}{0,0,0.4}
%%% Configuracion de Listings
\lstloadlanguages{C++}
\lstnewenvironment{code}
{%\lstset{	numbers=none, frame=lines, basicstyle=\small\ttfamily, }%
	\csname lst@SetFirstLabel\endcsname}
{\csname lst@SaveFirstLabel\endcsname}
\lstset{% general command to set parameter(s)
	language=C++, basicstyle=\small\ttfamily, keywordstyle=\slshape,
	emph=[1]{tipo,usa}, emphstyle={[1]\sffamily\bfseries},
	morekeywords={tint,forn,forsn},
	basewidth={0.47em,0.40em},
	columns=fixed, fontadjust, resetmargins, xrightmargin=5pt, xleftmargin=15pt,
	flexiblecolumns=false, tabsize=2, breaklines,	breakatwhitespace=false, extendedchars=true,
	numbers=left, numberstyle=\tiny, stepnumber=1, numbersep=9pt,
	frame=l, framesep=3pt,
	basicstyle=\ttfamily,
	keywordstyle=\color{darkblue}\ttfamily,
	stringstyle=\color{magenta}\ttfamily,
	commentstyle=\color{RedOrange}\ttfamily,
	morecomment=[l][\color{OliveGreen}]{\#}
}

\lstdefinestyle{C++}{
	language=C++, basicstyle=\small\ttfamily, keywordstyle=\slshape,
	emph=[1]{tipo,usa,tipo2}, emphstyle={[1]\sffamily\bfseries},
	morekeywords={tint,forn,forsn},
	basewidth={0.47em,0.40em},
	columns=fixed, fontadjust, resetmargins, xrightmargin=5pt, xleftmargin=15pt,
	flexiblecolumns=false, tabsize=2, breaklines,	breakatwhitespace=false, extendedchars=true,
	numbers=left, numberstyle=\tiny, stepnumber=1, numbersep=9pt,
	frame=l, framesep=3pt,
	basicstyle=\ttfamily,
	keywordstyle=\color{darkblue}\ttfamily,
	stringstyle=\color{magenta}\ttfamily,
	commentstyle=\color{RedOrange}\ttfamily,
	morecomment=[l][\color{OliveGreen}]{\#}
}


\lstloadlanguages{bash}

% Define a custom style for Bash code
\lstdefinestyle{bash}{
	language=bash, basicstyle=\small\ttfamily, keywordstyle=\slshape,
emph=[1]{tipo,usa,tipo2}, emphstyle={[1]\sffamily\bfseries},
morekeywords={tint,forn,forsn},
basewidth={0.47em,0.40em},
columns=fixed, fontadjust, resetmargins, xrightmargin=5pt, xleftmargin=15pt,
flexiblecolumns=false, tabsize=2, breaklines,	breakatwhitespace=false, extendedchars=true,
numbers=left, numberstyle=\tiny, stepnumber=1, numbersep=9pt,
frame=l, framesep=3pt,
basicstyle=\ttfamily,
keywordstyle=\color{darkblue}\ttfamily,
stringstyle=\color{magenta}\ttfamily,
commentstyle=\color{RedOrange}\ttfamily,
morecomment=[l][\color{OliveGreen}]{\#}
}

%%% Macros
\def\nbtitle#1{\begin{Large}\begin{center}\textbf{#1}\end{center}\end{Large}}
\def\nbsection#1{\section{#1}}
\def\nbsubsection#1{\subsection{#1}}
\def\nbcoment#1{\begin{small}\textbf{#1}\end{small}}
\newcommand{\comb}[2]{\left( \begin{array}{c} #1 \\ #2 \end{array}\right)}
\def\complexity#1{\texorpdfstring{$\mathcal{O}(#1)$}{O(#1)}}
\newcommand\cppfile[2][]{
	\lstinputlisting[style=C++]{#2}
}
\newcommand\bashfile[2][]{
	\lstinputlisting[style=bash]{#2}
}
\begin{document}
	\def\title{Equipo 1 Facultad de Ingenieria}
	\tableofcontents\newpage

  \section{Funciones C++}
  \input{funciones.tex}

	\section{Compile}
  \subsection{Compile}
\cppfile{../compile/compile.cpp}

\subsection{Template}
\cppfile{../compile/template.cpp}

  \section{Data Structures}
	\subsection{BIT}
\cppfile{../data_structures/bit.cpp}

\subsection{Bitset}
\cppfile{../data_structures/Bitset.cpp}

\subsection{Bit Trie}
\cppfile{../data_structures/BitTrie.cpp}

\subsection{CHMIN/CHMAX Rangesum GCD Queries}
\cppfile{../data_structures/chmin_chmax_rangesum_gcdqueries.cpp}

\subsection{Disjoint Sparse Table}
\cppfile{../data_structures/DisjointSparseTable.cpp}

\subsection{Disjoint Set Union Bipartite}
\cppfile{../data_structures/DSU_bipartite.cpp}

\subsection{Disjoint Set Union}
\cppfile{../data_structures/DSU.cpp}

\subsection{Disjoint Set Union Queue Like Rollback}
\cppfile{../data_structures/DSUQueueLikeRollback.cpp}

\subsection{Dynamic Conectivity}
\cppfile{../data_structures/DynamicConnectivity.cpp}

\subsection{Fenwick Tree}
\cppfile{../data_structures/Fenwick.cpp}

\subsection{Fenwick Tree 2D}
\cppfile{../data_structures/Fenwick2D.cpp}

\subsection{Li Chao Tree}
\cppfile{../data_structures/li_chao_tree.cpp}

\subsection{Link Cut Tree}
\cppfile{../data_structures/link_cut_tree.cpp}

\subsection{Linked List}
\cppfile{../data_structures/LinkedList.cpp}

\subsection{Merge Sort Tree}
\cppfile{../data_structures/MergeSortTree.cpp}

\subsection{Minimum Cartesian Tree}
\cppfile{../data_structures/min_cartesian_tree.cpp}

\subsection{Multi Ordered Set}
\cppfile{../data_structures/multioset.cpp}

\subsection{Ordered Set}
\cppfile{../data_structures/oset.cpp}

\subsection{Palindromic Tree}
\cppfile{../data_structures/palindromic_tree.cpp}

\subsection{Persistent Array}
\cppfile{../data_structures/PersistentArray.cpp}

\subsection{Persistent Segment Tree}
\cppfile{../data_structures/PersistentSegTree.cpp}

\subsection{Polynomial Queries}
\cppfile{../data_structures/polynomial_queries.cpp}

\subsection{Segment Tree}
\cppfile{../data_structures/SegTree.cpp}

\subsection{Segment Tree 2D}
\cppfile{../data_structures/SegTree2D.cpp}

\subsection{Segment Tree Dynamic}
\cppfile{../data_structures/SegTreeDynamic.cpp}

\subsection{Segment Tree Lazy Types}
\cppfile{../data_structures/SegTreeLazy_types.cpp}

\subsection{Segment Tree Lazy}
\cppfile{../data_structures/SegTreeLazy.cpp}

\subsection{Segment Tree Lazy Range Set}
\cppfile{../data_structures/SegTreeLazyRangeSet.cpp}

\subsection{Segment Tree Lazy Range Set 2}
\cppfile{../data_structures/SegTreeLazyRangeSet2.cpp}

\subsection{Segment Tree Max Subarray Sum}
\cppfile{../data_structures/SegTreeMaxSubarraySum.cpp}

\subsection{Segment Tree Range Update}
\cppfile{../data_structures/SegTreeRangeUpdate.cpp}

\subsection{Segment Tree Struct Types}
\cppfile{../data_structures/SegTreeStruct_types.cpp}

\subsection{Segment Tree Struct}
\cppfile{../data_structures/SegTreeStruct.cpp}

\subsection{Segment Tree Walk}
\cppfile{../data_structures/SegTreeWalk.cpp}

\subsection{Sparse Table}
\cppfile{../data_structures/SparseTable.cpp}

\subsection{Sparse Table 2D}
\cppfile{../data_structures/SparseTable2D.cpp}

\subsection{Square Root Decomposition}
\cppfile{../data_structures/SqrtDecomposition.cpp}

\subsection{Tuple}
\cppfile{../data_structures/tuple.cpp}

\subsection{Treap}
\cppfile{../data_structures/TREAP/treap.cpp}

\subsection{Treap 2}
\cppfile{../data_structures/TREAP/treap2.cpp}

\subsection{Treap With Inversion}
\cppfile{../data_structures/TREAP/treap_w_inversion.cpp}

\subsection{Treap With Inversions and Range Updates}
\cppfile{../data_structures/TREAP/treap_inversions_and_range_update.cpp}

  \section{Dynamic Programming}
  \subsection{CHT Deque}
\cppfile{../dp/CHTDeque.cpp}

\subsection{Digit DP}
\cppfile{../dp/DigitDP.cpp}

\subsection{Divide and Conquer DP}
\cppfile{../dp/divide_conquer_dp.cpp}

\subsection{Edit Distance}
\cppfile{../dp/EditDistance.cpp}

\subsection{Knuth's Algorithm}
\cppfile{../dp/Knuth.cpp}

\subsection{LCS}
\cppfile{../dp/LCS.cpp}

\subsection{Line Container}
\cppfile{../dp/LineContainer.cpp}

\subsection{Longest Increasing Subsequence}
\cppfile{../dp/LIS.cpp}

\subsection{SOS DP}
\cppfile{../dp/sosDp.cpp}


  \section{Flow}
  \subsection{Dinic}
\cppfile{../flow/dinic.cpp}

\subsection{Hopcroft-Karp}
\cppfile{../flow/hopcroft_karp.cpp}

\subsection{Hungarian}
\cppfile{../flow/hungarian.cpp}

\subsection{Max Flow Min Cost}
\cppfile{../flow/max_flow_min_cost.cpp}

\subsection{Max Flow}
\cppfile{../flow/max_flow.cpp}

\subsection{Min Cost Max Flow}
\cppfile{../flow/MinCostMaxFlow.cpp}

\subsection{Push Relabel}
\cppfile{../flow/push_relabel.cpp}

  \section{Geometry}
  \subsection{Point Struct}
\cppfile{../geometry/point.cpp}

\subsection{Sort Points}
\cppfile{../geometry/sort_points.cpp}

\subsection{Point Struct2}
\cppfile{../geometry/point_struct.cpp}

\subsection{Antipodal Pairs}
\cppfile{../geometry/antipodal_pairs.cpp}

\subsection{Area and Perimeter}
\cppfile{../geometry/area_perimeter.cpp}

\subsection{Area Union Circles}
\cppfile{../geometry/area_union_circle.cpp}

\subsection{Centroid}
\cppfile{../geometry/centroid.cpp}

\subsection{Circle Inside Circle}
\cppfile{../geometry/circle_inside_circle.cpp}

\subsection{Circle Outside Circle}
\cppfile{../geometry/circle_outside_circle.cpp}

\subsection{Closest Pair of Points}
\cppfile{../geometry/closest_pair_of_points.cpp}

\subsection{Common Tangents}
\cppfile{../geometry/common_tangents.cpp}

\subsection{Convex Hull}
\cppfile{../geometry/convex_hull.cpp}

\subsection{Segment Intersection with Ray}
\cppfile{../geometry/cross_ray.cpp}

\subsection{Cut Polygon}
\cppfile{../geometry/cut_polygon.cpp}

\subsection{Delaunay Triangulation}
\cppfile{../geometry/delaunay_triangulation.cpp}

\subsection{Diameter and Width}
\cppfile{../geometry/diameter_and_width.cpp}

\subsection{Distance Between Point and Circle}
\cppfile{../geometry/distance_point_circle.cpp}

\subsection{Distance Between Point and Line}
\cppfile{../geometry/distance_point_line.cpp}

\subsection{Example of Geometry}
\cppfile{../geometry/example.cpp}

\subsection{Half Plane Intersection}
\cppfile{../geometry/half_plane_intersection.cpp}

\subsection{Incircle}
\cppfile{../geometry/incircle.cpp}

\subsection{Intersection of Two Circles}
\cppfile{../geometry/intersect_circles.cpp}

\subsection{Intersection Line and Circle}
\cppfile{../geometry/intersect_line_circle.cpp}

\subsection{Intersection Polygon and Circle}
\cppfile{../geometry/intersect_polygon_circle.cpp}

\subsection{Intersection Segment and Circle}
\cppfile{../geometry/intersect_segment_circle.cpp}

\subsection{Line Intersection}
\cppfile{../geometry/line_intersection.cpp}

\subsection{Minkowski Sum}
\cppfile{../geometry/minkowski_sum.cpp}

\subsection{Point in Circle}
\cppfile{../geometry/point_in_circle.cpp}

\subsection{Point in Convex Hull}
\cppfile{../geometry/point_in_convex_hull.cpp}

\subsection{Point in Line}
\cppfile{../geometry/point_in_line.cpp}

\subsection{Point in Perimeter}
\cppfile{../geometry/point_in_perimeter.cpp}

\subsection{Point in Polygon}
\cppfile{../geometry/point_in_polygon.cpp}

\subsection{Point in Segment}
\cppfile{../geometry/point_in_segment.cpp}

\subsection{Points Tangency}
\cppfile{../geometry/points_tangency.cpp}

\subsection{Projection Point Circle}
\cppfile{../geometry/projection_point_circle.cpp}

\subsection{Segment Intersection}
\cppfile{../geometry/segment_intersection.cpp}

\subsection{Smallest Enclosing Circle}
\cppfile{../geometry/smallest_enclosing_circle.cpp}

\subsection{Smallest Enclosing Rectangle}
\cppfile{../geometry/smallest_enclosing_rectangle.cpp}

\subsection{Vantage Point Tree}
\cppfile{../geometry/vantage_point_tree.cpp}

  \section{Graphs}
  \subsection{2Sat}
\cppfile{../graph/2Sat.cpp}

\subsection{Articulation Points}
\cppfile{../graph/ArticulationPoints.cpp}

\subsection{Bellman-Ford}
\cppfile{../graph/BellmanFord.cpp}

\subsection{Bipartite Checker}
\cppfile{../graph/bipartite_checker.cpp}

\subsection{Bipartite Maximum Matching}
\cppfile{../graph/bipartite_maximum_matching.cpp}

\subsection{Block Cut Tree}
\cppfile{../graph/BlockCutTree.cpp}

\subsection{Blossom}
\cppfile{../graph/blossom.cpp}

\subsection{Bridges}
\cppfile{../graph/Bridges.cpp}

\subsection{Bridges Online}
\cppfile{../graph/BridgesOnline.cpp}

\subsection{Dijkstra}
\cppfile{../graph/Dijkstra.cpp}

\subsection{Eulerian Path}
An Eulerian Path is a path that passes through every edge once. For an undirected graph an eulerian path exists if the degree of every node is even or the degree of exactly two nodes is odd. In the first case, the eulerian path is also an eulerian circuit or cycle. In a directed graph, an eulerian path exists if at most one noda has $out_i-in_i=1$ and at most one node has $in_i-out_i=1$. A cycle exists if $in_i-out_i=0$ for all i. 
\cppfile{../graph/EulerianPath.cpp}

\subsection{Floyd-Warshall}
\cppfile{../graph/FloydWarshall.cpp}

\subsection{Kruskal}
\cppfile{../graph/Kruskal.cpp}

\subsection{Marriage}
\cppfile{../graph/Marriage.cpp}

\subsection{SCC}
\cppfile{../graph/scc.cpp}

  \section{Linear Algebra}
  \subsection{Simplex}
\cppfile{../linear_algebra/simplex.cpp}

  \section{Math}
  \subsection{BinPow}
\cppfile{../mathematics/Binpow.cpp}

\subsection{Combination Rank}
\cppfile{../mathematics/combinationRank.cpp}

\subsection{Diophantine}
If one solution is $(x_0, y_0)$ all solutions can be obtained by $x = x_0  +k  * \frac{b}{gcd(a, b)}$ and $y= y_0 -k *\frac{a}{gcd(a, b)}$.
\cppfile{../mathematics/diophantine.cpp}

\subsection{Discrete Logarithm}
Finds discrete logarithm in $O(\sqrt(m))$.
\cppfile{../mathematics/discrete_log.cpp}

\subsection{Divisors}
\cppfile{../mathematics/divisors.cpp}

\subsection{Euler Totient (Phi)}
\cppfile{../mathematics/EulerTotient.cpp}

\subsection{Fibonacci}
\cppfile{../mathematics/fib.cpp}

\subsection{Matrix Exponentiation}
\cppfile{../mathematics/matrix_exp.cpp}

\subsection{Miller Rabin Deterministic}
\cppfile{../mathematics/MillerRabinDeterministic.cpp}

\subsection{Mobius}
\cppfile{../mathematics/mobius.cpp}

\subsection{Permutation Rank}
\cppfile{../mathematics/permutationRank.cpp}

\subsection{Prefix Sum Phi}
\cppfile{../mathematics/prefix_sum_phi.cpp}

\subsection{Sieve}
\cppfile{../mathematics/sieve.cpp}
  \subsection{Identities}
	{
		$C_n = \frac{2(2n-1)}{n+1} C_{n-1}$

		$C_n = \frac{1}{n+1} \binom{2n}{n}$

		$C_n \sim \frac{4^n}{n^{3/2}\sqrt{\pi}}$

		$\sigma(n) = O(\log(\log(n)))$ (number of divisors of $n$)

		$F_{2n+1} = F_{n}^2 + F_{n+1}^2$

		$F_{2n} = F_{n+1}^2 - F_{n-1}^2$

		$\sum_{i=1}^n F_i = F_{n+2}-1$

		$F_{n+i}F_{n+j} - F_nF_{n+i+j} = (-1)^n F_iF_j$

		(Möbius Inv. Formula)
		$\mu(p^k) = [k=0] - [k=1]$
		Let $g(n) = \sum_{d\mid n} f(d)$, then $f(n)=\sum_{d\mid n} g(d) \mu\left(\frac{n}{d}\right)$.

		(Dirichlet Convolution)
		Let $f,g$ be arithmetic functions, then $(f*g)(n) = \sum_{d\mid n} f(d)g\left(\frac{n}{d}\right)$.
		If $f,g$ are multiplicative, then so is $f*g$.

		$n=\sum_{d|n}\phi(d)$

		Lucas' Theorem: $\binom{m}{n} \equiv \prod_{i=0}^k \binom{m_i}{n_i} \pmod{p}$ where $m = \sum_{i=0}^k m_i p^i$ and $n = \sum_{i=0}^k n_i p^i$.
	}
	\subsection{Burnside's Lemma}
		\begin{flushleft}
		Dado un grupo $G$ de permutaciones y un conjunto $X$ de $n$ elementos, 
		el número de órbitas de $X$ bajo la acción de $G$ es igual al promedio
		del número de puntos fijos de las permutaciones en $G$.\\
		Formalmente, el número de órbitas es $\frac{1}{|G|} \sum_{g \in G} f(g)$ donde $f(g)$ es el número de puntos fijos de $g$.\\
		Ejemplo: Dado un collar con $n$ cuentas y $2$ colores, el número de collares
		diferentes que se pueden formar es $\frac{1}{n} \sum_{i=0}^{n} f(i)$ donde $f(i)$ es el número de collares
		que quedan fijos bajo una rotación de $i$ posiciones.\\
		Para contar el número de collares que quedan fijos bajo una rotación de $i$ posiciones, se puede usar
		la fórmula $f(i) = 2^{\gcd(i,n)}$.\\
		Para un collar de $n$ cuentas y $k$ colores, el número de collares diferentes que se pueden formar
		es $\frac{1}{n} \sum_{i=0}^{n} k^{\gcd(i,n)}$\\
		Ejemplo: Dado un cubo con $6$ caras y $k$ colores, el número de cubos diferentes que se pueden formar
		es $\frac{1}{24} \sum_{i=0}^{24} f(i)$ donde $f(i)$ es el número de cubos que quedan fijos bajo una rotación de $i$ posiciones.
		Esta formula es igual a $\frac{1}{24} (n^6+3n^4+12n^3+8n^2)$
		\end{flushleft}
	\subsection{Recursion}
		\begin{flushleft}
			Sea $f(n) = \sum_{i=1}^{k} a_i f(n-i)$ entonces
			podemos considerar la matriz:
			\[
			\begin{bmatrix}
				f(n) \\
				f(n-1) \\
				\vdots \\
				f(n-k+1)
			\end{bmatrix}
			=
			\begin{bmatrix}
				a_1 & a_2 & \cdots & a_{k-1} & a_{k}\\
				1 & 0 & \cdots & 0 & 0\\
				0 & 1 & \cdots & 0 & 0\\
				\vdots & \vdots & \ddots & \vdots & \vdots\\
				0 & 0 & \cdots & 1 & 0
			\end{bmatrix}
			\begin{bmatrix}
				f(n-1) \\
				f(n-2) \\
				\vdots \\
				f(n-k)
			\end{bmatrix}
			\]
			De aqui podemos calcular $f(n)$ con exponenciación de matrices.
			\[
			\begin{bmatrix}
				f(n) \\
				f(n-1) \\
				\vdots \\
				f(n-k+1)
			\end{bmatrix}
			=
			\begin{bmatrix}
				a_1 & a_2 & \cdots & a_{k-1} & a_{k}\\
				1 & 0 & \cdots & 0 & 0\\
				0 & 1 & \cdots & 0 & 0\\
				\vdots & \vdots & \ddots & \vdots & \vdots\\
				0 & 0 & \cdots & 1 & 0
			\end{bmatrix}
			^{n-k}
			\begin{bmatrix}
				f(k) \\
				f(k-1) \\
				\vdots \\
				f(1)
			\end{bmatrix}
			\]
		\end{flushleft}
	\subsection{Theorems}
		\begin{flushleft}
			\textbf{Koeing's Theorem:} La cardinalidad del emparejamiento maximo de una grafica bipartita
			es igual al minimum vertex cover.\\
			\textbf{Hall's Theorem:} Una grafica bipartita $G$ tiene un emparejamiento que cubre todos los nodos
			de $G$ si y solo si para todo subconjunto $S$ de nodos de $G$, el número de vecinos de $S$ es mayor o igual a $|S|$.\\
			\textbf{Kuratowski's Theorem:} Una grafica es plana si y solo si no contiene un subgrafo homeomorfo a $K_{3,3}$ o $K_5$.\\
		\end{flushleft}

\subsection{Sums}
\[ c^a + c^{a+1} + \dots + c^{b} = \frac{c^{b+1} - c^a}{c-1}, c \neq 1 \]
\begin{align*}
	1 + 2 + 3 + \dots + n &= \frac{n(n+1)}{2} \\
	1^2 + 2^2 + 3^2 + \dots + n^2 &= \frac{n(2n+1)(n+1)}{6} \\
	1^3 + 2^3 + 3^3 + \dots + n^3 &= \frac{n^2(n+1)^2}{4} \\
	1^4 + 2^4 + 3^4 + \dots + n^4 &= \frac{n(n+1)(2n+1)(3n^2 + 3n - 1)}{30} \\
\end{align*}

\subsection{Catalan numbers}
\[ C_n=\frac{1}{n+1}\binom{2n}{n}= \binom{2n}{n}-\binom{2n}{n+1} = \frac{(2n)!}{(n+1)!n!} \]
\[ C_0=1,\ C_{n+1} = \frac{2(2n+1)}{n+2}C_n,\ C_{n+1}=\sum C_iC_{n-i} \]
${C_n = 1, 1, 2, 5, 14, 42, 132, 429, 1430, 4862, 16796, 58786, \dots}$
\begin{itemize}
	\item sub-diagonal monotone paths in an $n\times n$ grid.
	\item strings with $n$ pairs of parenthesis, correctly nested. If prefix is given, number of ways is $\binom{n}{remaining_closed}$-$\binom{n}{remaining_closed+1}$.
	\item binary trees with with $n+1$ leaves (0 or 2 children).
	\item ordered trees with $n+1$ vertices.
	\item ways a convex polygon with $n+2$ sides can be cut into triangles by connecting vertices with straight lines.
	\item permutations of $[n]$ with no 3-term increasing subseq.
\end{itemize}
\
\subsection{Cayley's formula}
Number of labeled trees of n vertices is $n^{n-2}$.
Number of rooted forest of n vertices is $(n+1)^{n-1}$.

\subsection{Geometric series}
Infinite \\
$a+ar+ar^2+ar^3+........+ \sum_{k=0}^{\infty} ar^k $	
\\
Sum=$\frac{a}{1-r}$
\\
Finite \\
$a+ar+ar^2+ar^3+........+ \sum_{k=0}^{n} ar^k $
\\
Sum=$\frac{a(1-r^{n+1})}{1-r}$


\subsection{Estimates For Divisors}

$\sum_{d|n} d = O(n \log \log n)$.

The number of divisors of $n$   is at most around 100 for $n < 5e4$, 500 for $n < 1e7$, 2000 for $n < 1e10$, 200\,000 for $n < 1e19$.

\subsection{Sum of divisors}

$\sum{d|n} =\frac{{p_1}^{\alpha_1+1}-1}{p_1-1}+\frac{{p_2}^{\alpha_2+1}-1}{p_2-1}+....+\frac{{p_n}^{\alpha_n+1}-1}{p_n-1}$

\subsection{Pythagorean Triplets}
The Pythagorean triples are uniquely generated by
\[ a=k\cdot (m^{2}-n^{2}),\ \,b=k\cdot (2mn),\ \,c=k\cdot (m^{2}+n^{2}), \]
with $m > n > 0$, $k > 0$, $m \bot n$, and either $m$ or $n$ even.


\subsection{Derangements}
Permutations of a set such that none of the elements appear in their original position.
\[ \mkern-2mu D(n) = (n-1)(D(n-1)+D(n-2)) = n D(n-1)+(-1)^n = \left\lfloor\frac{n!}{e}\right\rceil \]


\section{Game Theory}
	\subsection{Sprague-Grundy theorem}
	\url{https://codeforces.com/blog/entry/66040}
	Dado un juego con pilas $p_1$, $p_2$, ..., $p_n$ sea $g(p)$ el nimber de la pila $p$, 
	entonces el nimber del juego es $g(p_1) \oplus g(p_2) \oplus ... \oplus g(p_n)$.
	Para calcular el nimber de una pila, se puede usar la fórmula $g(r) = mex(\{g(r_1), g(r_2), ..., g(r_k)\})$
	donde $r_1, r_2, ..., r_k$ son los posibles estados a los que se puede llegar desde $r$ y $g(r)=0$ si $r$ es un estado perdedor.


	\section{Fórmulas y notas}
	\subsection{Números de Stirling del primer tipo}
		$\stirlingI{n}{k}$ representa el número de permutaciones de $n$ elementos en exactamente $k$ ciclos disjuntos.
		\begin{align*}
			\stirlingI{0}{0} &= 1 \\
			\stirlingI{0}{n} &= \stirlingI{n}{0} = 0 \quad &, \quad n>0 \\
			\stirlingI{n}{k} &= (n-1)\stirlingI{n-1}{k} + \stirlingI{n-1}{k-1} \quad &, \quad k>0 \\
			\sum_{k=0}^{n} \stirlingI{n}{k} &= n! \\
			\sum_{k=0}^{\infty} \stirlingI{n}{k} x^k &= \prod_{k=0}^{n-1}(x+k)
		\end{align*}
	
	\subsection{Números de Stirling del segundo tipo}
		$\stirlingII{n}{k}$ representa el número de formas de particionar un conjunto de $n$ objetos distinguibles en $k$ subconjuntos no vacíos.
		\begin{align*}
			\stirlingII{0}{0} &= 1 \\
			\stirlingII{0}{n} &= \stirlingII{n}{0} = 0 \quad &, \quad n>0 \\
			\stirlingII{n}{k} &= k\stirlingII{n-1}{k} + \stirlingII{n-1}{k-1} \quad &, \quad k>0 \\
			&= \sum_{j=0}^{k} \dfrac{j^n}{j!} \cdot \dfrac{(-1)^{k-j}}{(k-j)!}
		\end{align*}
	
	\subsection{Números de Euler}
		$\euler{n}{k}$ representa el número de permutaciones de $1$ a $n$ en donde exactamente $k$ números son mayores que el número anterior, es decir, cuántas permutaciones tienen $k$ ``ascensos''.
		\begin{align*}
			\euler{1}{0} &= 1 \\
			\euler{n}{k} &= (n-k)\euler{n-1}{k-1} + (k+1)\euler{n-1}{k} \quad &, \quad n \geq 2 \\
			&= \sum_{j=0}^{k} (-1)^j \binom{n+1}{j} (k+1-j)^n \\
			\sum_{k=0}^{n-1} \euler{n}{k} &= n!
		\end{align*}
	
	\subsection{Números de Catalan}
		\begin{align*}
			C_0 &= 1 \\
			C_n &= \dfrac{1}{n+1}\binom{2n}{n} = \sum_{j=0}^{n-1} C_j C_{n-1-j} \\
			\sum_{n=0}^{\infty} C_n x^n &= \dfrac{1-\sqrt{1-4x}}{2x}
		\end{align*}
	
	\subsection{Números de Bell}
		$B_n$ representa el número de formas de particionar un conjunto de $n$ elementos.
		\begin{align*}
			B_n &= \sum_{k=0}^{n}\stirlingII{n}{k} = \sum_{k=0}^{n-1}\binom{n-1}{k} B_k \\
			\sum_{n=0}^{\infty} \dfrac{B_n}{n!}x^n &= e^{e^x-1}
		\end{align*}
	
	\subsection{Números de Bernoulli}
		\begin{align*}
			{B_0}^+ &= 1 \\
			{B_n}^+ &= 1 - \sum_{k=0}^{n-1}\binom{n}{k}\dfrac{{B_k}^+}{n-k+1} \\
			\sum_{n=0}^{\infty} \dfrac{{B_n}^+ x^n}{n!} &= \dfrac{x}{1-e^{-x}} = \dfrac{1}{\frac{1}{1!}-\frac{x}{2!}+\frac{x^2}{3!}-\frac{x^3}{4!}+\cdots}
		\end{align*}
	
	\subsection{Fórmula de Faulhaber}
		\begin{align*}
			S_p(n) &= \sum_{k=1}^{n}k^p = \dfrac{1}{p+1}\sum_{k=0}^{p} \binom{p+1}{k} {B_k}^+ n^{p+1-k}
		\end{align*}
	
	\subsection{Función Beta}
		\begin{align*}
			B(x,y) &= \dfrac{\Gamma(x)\Gamma(y)}{\Gamma(x+y)} = 2 \int_{0}^{\pi/2} \sin^{2x-1}(\theta) \cos^{2x-1}(\theta) d\theta \\
			&= \int_{0}^{1} t^{x-1} (1-t)^{y-1} dt = \int_{0}^{\infty} \dfrac{t^{x-1}}{(1+t)^{x+y}} dt
		\end{align*}
		
	\subsection{Función zeta de Riemann}
		La siguiente fórmula converge rápido para valores pequeños de $n$ ($n \approx 20$):
		\begin{align*}
			\zeta(s) &\approx \dfrac{1}{d_0 (1 - 2^{1-s})} \sum_{k=1}^{n} \dfrac{(-1)^{k-1} d_k}{k^s} \\
			d_k &= \sum_{j=k}^{n} \dfrac{4^j}{n+j} \binom{n+j}{2j}
		\end{align*}
	
	\subsection{Funciones generadoras}
		\begin{align*}
			\sum_{n=0}^{\infty} \left( \sum_{k=0}^{n}a_k \right) x^n &= \dfrac{1}{1-x}\sum_{n=0}^{\infty} a_n x^n \\
			\sum_{n=0}^{\infty} \binom{n+k-1}{k-1}x^n &= \dfrac{1}{\left(1-x\right)^k} \\
			\sum_{n=0}^{\infty} p_n x^n &= \dfrac{1}{\displaystyle \prod_{k=1}^{\infty}(1-x^k)} = \dfrac{1}{\displaystyle \sum_{n=-\infty}^{\infty} (-1)^n x^{\frac{1}{2}n(3n+1)}} \\
			\sum_{p=0}^{\infty} \dfrac{S_p(n)}{p!} x^p &= \dfrac{e^{x(n+1)}-e^x}{e^x-1} \\
			\sum_{n=0}^{\infty} n^k x^n &= \dfrac{\displaystyle \sum_{i=0}^{k-1} \euler{k}{i} x^{i+1}}{(1-x)^{k+1}} \quad , \quad k \geq 1
		\end{align*}
		Sean $a_1, a_2, \ldots, a_n$ números complejos. Sean $p_m = \displaystyle \sum_{i=1}^{n} a_i^m$ y $s_m$ el $m$-ésimo polinomio elemental simétrico de $a_1, a_2, \ldots, a_n$. Entonces se cumple que $xS'(x) + P(x)S(x) = 0$, donde $P(x)=\displaystyle \sum_{m=1}^{\infty} p_m x^m$ y $S(x)=\displaystyle \prod_{i=1}^{n}(1-a_ix) = \sum_{m=0}^{n}(-1)^m s_m x^m$.
	
	\subsection{Números armónicos}
		\begin{align*}
			H_n &= \sum_{k=1}^{n} \dfrac{1}{k} \approx \ln(n) + \gamma + \dfrac{1}{2n} - \dfrac{1}{12n^2} \\
			\gamma &\approx 0.577215664901532860606512
		\end{align*}
	
	\subsection{Aproximación de Stirling}
		\begin{align*}
			\ln(n!) &\approx n\ln(n) - n + \dfrac{1}{2}\ln(2 \pi n) \\
			\text{\# de dígitos de $n!$} &= 1 + \left\lfloor n\log\left(\dfrac{n}{e}\right) + \dfrac{1}{2}\log(2 \pi n) \right\rfloor \quad \text{($n \geq 30$)}
		\end{align*}
	
	\subsection{Ternas pitagóricas}
		\begin{itemize}
			\item Una terna de enteros positivos $(a,b,c)$ es pitagórica si $a^2+b^2=c^2$. Además es primitiva si $\gcd(a,b,c)=1$.
			\item Generador de ternas primitivas:
			\begin{align*}
				a &= m^2-n^2 \\
				b &= 2mn \\
				c &= m^2+n^2
			\end{align*}
			donde $n \geq 1$, $m>n$, $\gcd(m,n)=1$ y $m,n$ tienen distinta paridad.
			\item Árbol de ternas pitagóricas primitivas: al multiplicar cualquiera de estas matrices:
			\begin{align*}
				\begin{pmatrix}
					1 & -2 & 2 \\
					2 & -1 & 2 \\
					2 & -2 & 3
				\end{pmatrix} \quad , \quad
				\begin{pmatrix}
					-1 & 2 & 2 \\
					-2 & 1 & 2 \\
					-2 & 2 & 3
				\end{pmatrix} \quad , \quad
				\begin{pmatrix}
					1 & 2 & 2 \\
					2 & 1 & 2 \\
					2 & 2 & 3
				\end{pmatrix}
			\end{align*}
			por una terna primitiva $\mathbf{v^T}$, obtenemos otra terna primitiva diferente. En particular, si empezamos con $\mathbf{v}=(3,4,5)$, podremos generar todas las ternas primitivas.
		\end{itemize}

	\subsection{Árbol de Stern–Brocot}
		Todos los racionales positivos se pueden representar como un árbol binario de búsqueda completo infinito con raíz $\frac{1}{1}$.
		\begin{itemize}
			\item Dado un racional $q=[a_0;a_1,a_2,\ldots,a_k]$ donde $a_k \neq 1$, sus hijos serán $[a_0;a_1,a_2,\ldots,a_k+1]$ y $[a_0;a_1,a_2,\ldots,a_k-1,2]$, y su padre será $[a_0;a_1,a_2,\ldots,a_k-1]$.
			\item Para hallar el camino de la raíz $\frac{1}{1}$ a un racional $q$, se usa búsqueda binaria iniciando con $L=\frac{0}{1}$ y $R=\frac{1}{0}$. Para hallar $M$ se supone que $L=\frac{a}{b}$ y $R=\frac{c}{d}$, entonces $M=\frac{a+c}{b+d}$.
		\end{itemize}

	\subsection{Combinatoria}
		\begin{itemize}
			\item Principio de las casillas: al colocar $n$ objetos en $k$ lugares hay al menos $\lceil \frac{n}{k} \rceil$ objetos en un mismo lugar.
			\item Número de funciones: sean $A$ y $B$ conjuntos con $m=\abs{A}$ y $n=\abs{B}$. Sea $f : A \to B$:
			\begin{itemize}
				\item Si $m \leq n$, entonces hay $\displaystyle m!\binom{n}{m}$ funciones inyectivas $f$.
				\item Si $m=n$, entonces hay $n!$ funciones biyectivas $f$.
				\item Si $m \geq n$, entonces hay $n!\stirlingII{m}{n}$ funciones suprayectivas $f$.
			\end{itemize}
			\item Barras y estrellas: ¿cuántas soluciones en los enteros no negativos tiene la ecuación $\displaystyle \sum_{i=1}^{k}x_i = n$? Tiene  $\displaystyle \binom{n+k-1}{k-1}$ soluciones.
			\item ¿Cuántas soluciones en los enteros positivos tiene la ecuación $\displaystyle \sum_{i=1}^{k}x_i = n$? Tiene  $\displaystyle \binom{n-1}{k-1}$ soluciones.
			\item Desordenamientos: $a_0=1$, $a_1=0$, $a_n=(n-1)(a_{n-1}+a_{n-2})=na_{n-1}+(-1)^n$.
			\item Sea $f(x)$ una función. Sea $g_n(x)=x g_{n-1}'(x)$ con $g_0(x)=f(x)$. Entonces $g_n(x)=\sum_{k=0}^{n} \stirlingII{n}{k} x^k f^{(k)}(x)$.
			\item Supongamos que tenemos $m+1$ puntos: $(0, y_0)$, $(1, y_1)$, $\ldots$, $(m, y_m)$. Entonces el polinomio $P(x)$ de grado $m$ que pasa por todos ellos es:
			\begin{align*}
				P(x) &= \left[ \prod_{i=0}^{m}(x-i) \right] (-1)^m \sum_{i=0}^{m} \dfrac{y_i (-1)^i}{(x-i)i!(m-i)!}
			\end{align*}
			\item Sea $a_0, a_1, \ldots$ una recurrencia lineal homogénea de grado $d$ dada por $\displaystyle a_n=\sum_{i=1}^{d} b_i a_{n-i}$ para $n \geq d$ con términos iniciales $a_0, a_1, \ldots, a_{d-1}$. Sean $A(x)$ y $B(x)$ las funciones generadoras de las sucesiones $a_n$ y $b_n$ respectivamente, entonces se cumple que $A(x)=\dfrac{A_0(x)}{1-B(x)}$, donde $\displaystyle A_0(x)=\sum_{i=0}^{d-1} \left[ a_i - \sum_{j=0}^{i-1}a_j b_{i-j} \right] x^i$.
			\item Si queremos obtener otra recurrencia $c_n$ tal que $c_n=a_{kn}$, las raíces del polinomio característico de $c_n$ se obtienen al elevar todas las raíces del polinomio característico de $a_n$ a la $k$-ésima potencia; y sus términos iniciales serán $a_0, a_k, \ldots, a_{k(d-1)}$.
		\end{itemize}

	\subsection{Grafos}
		\begin{itemize}
			\item Sea $d_n$ el número de grafos con $n$ vértices etiquetados: $\displaystyle d_n = 2^{\binom{n}{2}}$.
			\item Sea $c_n$ el número de grafos conexos con $n$ vértices etiquetados. Tenemos la recurrencia: $c_1=1$ y $\displaystyle d_n = \sum_{k=1}^{n} \binom{n-1}{k-1} c_k d_{n-k}$. También se cumple, usando funciones generadoras exponenciales, que $C(x)=1+\ln(D(x))$.
			\item Sea $t_n$ el número de torneos fuertemente conexos en $n$ nodos etiquetados. Tenemos la recurrencia $t_1=1$ y $\displaystyle d_n = \sum_{k=1}^{n} \binom{n}{k} t_k d_{n-k}$. Usando funciones generadoras exponenciales, tenemos que $T(x)=1-\dfrac{1}{D(x)}$.
			\item Número de spanning trees en un grafo completo con $n$ vértices etiquetados: $n^{n-2}$.
			\item Número de bosques etiquetados con $n$ vértices y $k$ componentes conexas: $kn^{n-k-1}$.
			\item Para un grafo no dirigido simple $G$ con $n$ vértices etiquetados de $1$ a $n$, sea $Q=D-A$, donde $D$ es la matriz diagonal de los grados de cada nodo de $G$ y $A$ es la matriz de adyacencia de $G$. Entonces el número de spanning trees de $G$ es igual a cualquier cofactor de $Q$.
			\item Sea $G$ un grafo. Se define al polinomio $P_G(x)$ como el polinomio cromático de $G$, en donde $P_G(k)$ nos dice cuántas $k$-coloraciones de los vértices admite $G$. Ejemplos comunes:
			\begin{itemize}
				\item Grafo completo de $n$ nodos: $P(x)=x(x-1)(x-2) \ldots (x-(n-1))$
				\item Grafo vacío de $n$ nodos: $P(x)=x^n$
				\item Árbol de $n$ nodos: $P(x)=x(x-1)^{n-1}$
				\item Ciclo de $n$ nodos: $P(x)=(x-1)^n + (-1)^n(x-1)$
			\end{itemize}
		\end{itemize}
	
	\subsection{Teoría de números}
		\begin{align*}
			(f * e)(n) &= f(n) \\
			(\varphi * \mathbf{1})(n) &= n \\
			(\mu * \mathbf{1})(n) &= e(n) \\
			\varphi(n^k) &= n^{k-1}\varphi(n) \\
			\sum_{\substack{k=1	\\ \gcd(k,n)=1}}^{n} k &= \dfrac{n \varphi(n)}{2} \quad , \quad n \geq 2 \\
			\sum_{k=1}^{n} \text{lcm}(k,n) &= \dfrac{n}{2} + \dfrac{n}{2}\sum_{d | n} d\varphi(d) = \dfrac{n}{2} + \dfrac{n}{2} \prod_{p^a | n} \dfrac{p^{2a+1}+1}{p+1} \\
			\sum_{k=1}^{n} \gcd(k,n) &= \sum_{d | n} d\varphi\left(\dfrac{n}{d}\right) = \prod_{p^a | n} p^{a-1}(1+(a+1)(p-1))
		\end{align*}
	
		\begin{itemize}
			\item Lifting the exponent: sea $p$ un primo, $x,y$ enteros y $n$ un entero positivo tal que $p \mid x-y$ pero $p \nmid x$ ni $p \nmid y$. Entonces:
			\begin{itemize}
				\item Si $p$ es impar: $v_p(x^n-y^n) = v_p(x-y) + v_p(n)$
				\item Si $p=2$ y $n$ es par: $v_p(x^n-y^n) = v_p(x-y) + v_p(n) + v_p(x+y) - 1$
			\end{itemize}
			donde $v_p(n)$ es el exponente de $p$ en la factorización en primos de $n$.
			\item Suma de dos cuadrados: sea $\chi_4(n)$ una función multiplicativa igual a 1 si $n \equiv 1 \mod 4$, $-1$ si $n \equiv 3 \mod 4$ y cero en otro caso. Entonces, el número de soluciones enteras $(a,b)$ de la ecuación $a^2+b^2=n$ es $4(\chi_4 * 1)(n) = 4 \displaystyle \sum_{d | n} \chi_4(d)$.
			\item Teorema de Lucas:
			\begin{align*}
				\binom{m}{n} &\equiv \prod_{i=0}^{k} \binom{m_i}{k_i} \pmod{p} \\
				m = \sum_{i=0}^{k} m_i p^i \quad &, \quad n = \sum_{i=0}^{k} n_i p^i \\
				0 \leq m_i &, n_i < p
			\end{align*}
			
			\item Sean $a,b,c \in \mathbb{Z}$ con $a \neq 0$ y $b \neq 0$. La ecuación $ax+by=c$ tiene como soluciones:
			\begin{align*}
				x &= \dfrac{x_0 c - bk}{d} \\
				y &= \dfrac{y_0 c + ak}{d} 
			\end{align*}
			para toda $k \in \mathbb{Z}$ si y solo si $d | c$, donde $ax_0+by_0=\gcd(a,b)=d$ (Euclides extendido). Si $a$ y $b$ tienen el mismo signo, hay exactamente $\max\left( \left\lfloor\dfrac{x_0 c}{\abs{b}}\right\rfloor + \left\lfloor\dfrac{y_0 c}{\abs{a}}\right\rfloor + 1, 0 \right)$ soluciones no negativas. Si tienen el signo distinto, hay infinitas soluciones no negativas.
			
			\item Dada una función aritmética $f$ con $f(1) \neq 0$, existe otra función aritmética $g$ tal que $(f*g)(n)=e(n)$, dada por:
			\begin{align*}
				g(1) &= \dfrac{1}{f(1)} \\
				g(n) &= -\dfrac{1}{f(1)} \sum_{d | n, d<n} f\left(\dfrac{n}{d}\right)g(d) \quad , \quad n > 1
			\end{align*}
			
			\item Sean $\displaystyle h(n) = \sum_{k=1}^{n} f\left(\left\lfloor \dfrac{n}{k} \right\rfloor\right) g(k)$, $\displaystyle G(n)=\sum_{k=1}^{n}g(k)$ y $m=\left\lfloor \sqrt{n} \right\rfloor$, entonces:
			\begin{align*}
				h(n) &= \sum_{k=1}^{\lfloor n/m \rfloor}f\left(\left\lfloor \dfrac{n}{k} \right\rfloor\right) g(k) + \sum_{k=1}^{m-1} \left( G\left(\left\lfloor \dfrac{n}{k} \right\rfloor\right) - G\left(\left\lfloor \dfrac{n}{k+1} \right\rfloor\right) \right)f(k)
			\end{align*}
			
			\item Sean $\displaystyle F(n)=\sum_{k=1}^{n}f(k)$, $\displaystyle G(n)=\sum_{k=1}^{n}g(k)$, $\displaystyle h(n)=(f * g)(n)=\sum_{d | n}f(d)g\left(\dfrac{n}{d}\right)$ y $\displaystyle H(n)=\sum_{k=1}^{n}h(k)$, entonces:
			\begin{align*}
				H(n) &= \sum_{k=1}^{n}f(k)G\left(\left\lfloor \dfrac{n}{k} \right\rfloor\right)
			\end{align*}
			
			\item Sean $\displaystyle \Phi_p(n) = \sum_{k=1}^{n}k^p\varphi(k)$ y $\displaystyle M_p(n) = \sum_{k=1}^{n}k^p\mu(k)$. Aplicando lo anterior, podemos calcular $\Phi_p(n)$ y $M_p(n)$ con complejidad $O(n^{2/3})$ si precalculamos con fuerza bruta los primeros $\lfloor n^{2/3} \rfloor$ valores, y para los demás, usamos las siguientes recurrencias (DP con \texttt{map}):
			{\small
			\begin{align*}
				\Phi_p(n) &= S_{p+1}(n) - \sum_{k=2}^{\lfloor n/m \rfloor} k^p \Phi_p\left(\left\lfloor \dfrac{n}{k}  \right\rfloor\right) - \sum_{k=1}^{m-1} \left( S_p\left(\left\lfloor \dfrac{n}{k} \right\rfloor\right) - S_p\left(\left\lfloor \dfrac{n}{k+1} \right\rfloor\right) \right)\Phi_p(k) \\
				M_p(n) &= 1 - \sum_{k=2}^{\lfloor n/m \rfloor} k^p M_p\left(\left\lfloor \dfrac{n}{k}  \right\rfloor\right) - \sum_{k=1}^{m-1} \left( S_p\left(\left\lfloor \dfrac{n}{k} \right\rfloor\right) - S_p\left(\left\lfloor \dfrac{n}{k+1} \right\rfloor\right) \right)M_p(k)
			\end{align*}
			}
			
			\item En general, si queremos hallar $F(n)$ y existe una función mágica $g(n)$ tal que $G(n)$ y $H(n)$ se puedan calcular en $O(1)$, entonces:
			{\small
			\begin{align*}
				F(n) &= \dfrac{1}{g(1)} \left[ H(n) - \sum_{k=2}^{\lfloor n/m \rfloor} g(k)F\left(\left\lfloor \dfrac{n}{k} \right\rfloor\right) - \sum_{k=1}^{m-1} \left( G\left(\left\lfloor \dfrac{n}{k} \right\rfloor\right) - G\left(\left\lfloor \dfrac{n}{k+1} \right\rfloor\right) \right)F(k) \right]
			\end{align*}
			}
		\end{itemize}
		
	\subsection{Primos}
		$10^2+1$, $10^3+9$, $10^4+7$, $10^5+3$, $10^6+3$, $10^7+19$, $10^8+7$, $10^9+7$, $10^{10}+19$, $10^{11}+3$, $10^{12}+39$, $10^{13}+37$, $10^{14}+31$, $10^{15}+37$, $10^{16}+61$, $10^{17}+3$, $10^{18}+3$.
		
		$10^2-3$, $10^3-3$, $10^4-27$, $10^5-9$, $10^6-17$, $10^7-9$, $10^8-11$, $10^9-63$, $10^{10}-33$, $10^{11}-23$, $10^{12}-11$, $10^{13}-29$, $10^{14}-27$, $10^{15}-11$, $10^{16}-63$, $10^{17}-3$, $10^{18}-11$.
	
	\subsection{Números primos de Mersenne}
		Números primos de la forma $M_p=2^p-1$ con $p$ primo. Todos los números perfectos pares son de la forma $2^{p-1}M_p$ y viceversa.
	
		Los primeros 47 valores de $p$ son: 2, 3, 5, 7, 13, 17, 19, 31, 61, 89, 107, 127, 521, 607, 1279, 2203, 2281, 3217, 4253, 4423, 9689, 9941, 11213, 19937, 21701, 23209, 44497, 86243, 110503, 132049, 216091, 756839, 859433, 1257787, 1398269, 2976221, 3021377, 6972593, 13466917, 20996011, 24036583, 25964951, 30402457, 32582657, 37156667, 42643801, 43112609.
	
	\subsection{Números primos de Fermat}
		Números primos de la forma $F_p=2^{2^p}+1$, solo se conocen cinco: 3, 5, 17, 257, 65537. Un polígono de $n$ lados es construible si y solo si $n$ es el producto de algunas potencias de dos y distintos primos de Fermat.


  \section{More Topics}
  \subsection{2D Prefix Sum}
\cppfile{../more_topics/2DPrefixSum.cpp}

\subsection{CDQ Divide and Conquer}
\cppfile{../more_topics/cdq_divide_and_conquer.cpp}

\subsection{Count Segments Around Index}
\cppfile{../more_topics/count_segments_around_index.cpp}

\subsection{Custom Comparators}
\cppfile{../more_topics/CustomComparators.cpp}

\subsection{Day of the Week}
\cppfile{../more_topics/day_of_week.cpp}

\subsection{Directed MST}
\cppfile{../more_topics/DirectedMST.cpp}

\subsection{GCD Convolution}
\cppfile{../more_topics/gcd_conv.cpp}

\subsection{int128}
\cppfile{../more_topics/int128.cpp}

\subsection{Iterating Over All Subsets}
\cppfile{../more_topics/iterating_over_all_subsets.cpp}

\subsection{LCM Convolution}
\cppfile{../more_topics/lcm_conv.cpp}

\subsection{Manhattan MST}
\cppfile{../more_topics/manhattan_mst.cpp}

\subsection{Max Manhattan Distance}
\cppfile{../more_topics/MaxManhattanDist.cpp}

\subsection{Mo}
\cppfile{../more_topics/mo.cpp}

\subsection{MOD INT}
\cppfile{../more_topics/mod_int.cpp}

\subsection{Next Permutation}
\cppfile{../more_topics/next_permutation.cpp}

\subsection{Next and Previous Smaller/Greater Element}
\cppfile{../more_topics/NextAndPrevSmaller_Greater.cpp}

\subsection{Parallel Binary Search}
\cppfile{../more_topics/ParallelBinarySearch.cpp}

\subsection{Random Number Generators}
\cppfile{../more_topics/randomNumberGenerators.cpp}

\subsection{setprecision}
\cppfile{../more_topics/setprecision.cpp}

\subsection{Ternary Search}
\cppfile{../more_topics/TernarySearch.cpp}

\subsection{Ternary Search Int}
\cppfile{../more_topics/TernarySearchInt.cpp}

\subsection{XOR Convolution}
\cppfile{../more_topics/xor_conv.cpp}

\subsection{XOR Basis}
\cppfile{../more_topics/XorBasis.cpp}

\subsection{XOR Basis Online}
\cppfile{../more_topics/XorBasisOnline.cpp}

  \section{Polynomials}
  \subsection{Berlekamp Massey}
\cppfile{../polynomials/berlekamp_massey.cpp}

\subsection{Binary Exp FFT}
\cppfile{../polynomials/binary_exp_fft.cpp}

\subsection{FFT}
\cppfile{../polynomials/FFT.cpp}

\subsection{NTT}
\cppfile{../polynomials/NTT.cpp}

\subsection{Roots NTT}
\cppfile{../polynomials/roots_ntt.txt}

  \section{Strings}
  \subsection{Hashed String}
\cppfile{../strings/HashedString.cpp}

\subsection{KMP}
\cppfile{../strings/KMP.cpp}

\subsection{Least Rotation String}
\cppfile{../strings/LeastRotationString.cpp}

\subsection{Manacher}
\cppfile{../strings/manacher.cpp}

\subsection{Suffix Array}
\cppfile{../strings/SuffixArray.cpp}

\subsection{Suffix Automaton}
\cppfile{../strings/SuffixAutomaton.cpp}

\subsection{Trie Ahocorasick}
\cppfile{../strings/TrieAho.cpp}

\subsection{Z Function}
\cppfile{../strings/z_function.cpp}

  \section{Trees}
  \subsection{Centroid Decomposition}
\cppfile{../trees/CentroidDecomposition.cpp}

\subsection{Heavy Light Decomposition}
\cppfile{../trees/HLD.cpp}

\subsection{Lowest Common Ancestor (LCA)}
\cppfile{../trees/LCA.cpp}

\subsection{Tree Diameter}
\cppfile{../trees/TreeDiameter.cpp}

  \section{Scripts}
  \subsection{build.sh}

This file should be called before stress.sh or validate.sh.
build.sh name.cpp

\bashfile{../scripts/build.sh}

\subsection{stress.sh}

Format is stress.sh Awrong Aslow Agen Numtests

\bashfile{../scripts/stress.sh}

\subsection{validate.sh}

Format is validate.sh Awrong Avalidator Agen NumTests

\bashfile{../scripts/validate.sh}

  \begin{table}[h!]
  \centering
  \begin{tabular}{|c|p{6cm}|c|}
  \hline
  \textbf{specifier} & \textbf{Output} & \textbf{Example} \\ \hline

  d or i & Signed decimal integer & 392 \\ \hline
  u & Unsigned decimal integer & 7235 \\ \hline
  o & Unsigned octal & 610 \\ \hline
  x & Unsigned hexadecimal integer (lowercase) & 7fa \\ \hline
  X & Unsigned hexadecimal integer (uppercase) & 7FA \\ \hline
  f & Decimal floating point, lowercase & 392.65 \\ \hline
  F & Decimal floating point, uppercase & 392.65 \\ \hline
  e & Scientific notation (mantissa/exponent), lowercase & 3.9265e+2 \\ \hline
  E & Scientific notation (mantissa/exponent), uppercase & 3.9265E+2 \\ \hline
  g & Use the shortest representation: \%e or \%f & 392.65 \\ \hline
  G & Use the shortest representation: \%E or \%F & 392.65 \\ \hline
  a & Hexadecimal floating point, lowercase & -0xc.90fep-2 \\ \hline
  A & Hexadecimal floating point, uppercase & -0XC.90FEP-2 \\ \hline
  c & Character & a \\ \hline
  s & String of characters & sample \\ \hline
  p & Pointer address & b8000000 \\ \hline
  n & Nothing printed. The argument must be a pointer to a \texttt{signed int}. The number of characters written so far is stored. & \\ \hline
  \% & A \% followed by another \% prints a single \% & \% \\ \hline

  \end{tabular}
  \end{table}

\end{document}