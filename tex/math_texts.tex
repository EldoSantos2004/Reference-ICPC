\subsection{Identities}
	{
		$C_n = \frac{2(2n-1)}{n+1} C_{n-1}$

		$C_n = \frac{1}{n+1} \binom{2n}{n}$

		$C_n \sim \frac{4^n}{n^{3/2}\sqrt{\pi}}$

		$\sigma(n) = O(\log(\log(n)))$ (number of divisors of $n$)

		$F_{2n+1} = F_{n}^2 + F_{n+1}^2$

		$F_{2n} = F_{n+1}^2 - F_{n-1}^2$

		$\sum_{i=1}^n F_i = F_{n+2}-1$

		$F_{n+i}F_{n+j} - F_nF_{n+i+j} = (-1)^n F_iF_j$

		(Möbius Inv. Formula)
		$\mu(p^k) = [k=0] - [k=1]$
		Let $g(n) = \sum_{d\mid n} f(d)$, then $f(n)=\sum_{d\mid n} g(d) \mu\left(\frac{n}{d}\right)$.

		(Dirichlet Convolution)
		Let $f,g$ be arithmetic functions, then $(f*g)(n) = \sum_{d\mid n} f(d)g\left(\frac{n}{d}\right)$.
		If $f,g$ are multiplicative, then so is $f*g$.

		$n=\sum_{d|n}\phi(d)$

		Lucas' Theorem: $\binom{m}{n} \equiv \prod_{i=0}^k \binom{m_i}{n_i} \pmod{p}$ where $m = \sum_{i=0}^k m_i p^i$ and $n = \sum_{i=0}^k n_i p^i$.
	}
	\subsection{Burnside's Lemma}
		\begin{flushleft}
		Dado un grupo $G$ de permutaciones y un conjunto $X$ de $n$ elementos, 
		el número de órbitas de $X$ bajo la acción de $G$ es igual al promedio
		del número de puntos fijos de las permutaciones en $G$.\\
		Formalmente, el número de órbitas es $\frac{1}{|G|} \sum_{g \in G} f(g)$ donde $f(g)$ es el número de puntos fijos de $g$.\\
		Ejemplo: Dado un collar con $n$ cuentas y $2$ colores, el número de collares
		diferentes que se pueden formar es $\frac{1}{n} \sum_{i=0}^{n} f(i)$ donde $f(i)$ es el número de collares
		que quedan fijos bajo una rotación de $i$ posiciones.\\
		Para contar el número de collares que quedan fijos bajo una rotación de $i$ posiciones, se puede usar
		la fórmula $f(i) = 2^{\gcd(i,n)}$.\\
		Para un collar de $n$ cuentas y $k$ colores, el número de collares diferentes que se pueden formar
		es $\frac{1}{n} \sum_{i=0}^{n} k^{\gcd(i,n)}$\\
		Ejemplo: Dado un cubo con $6$ caras y $k$ colores, el número de cubos diferentes que se pueden formar
		es $\frac{1}{24} \sum_{i=0}^{24} f(i)$ donde $f(i)$ es el número de cubos que quedan fijos bajo una rotación de $i$ posiciones.
		Esta formula es igual a $\frac{1}{24} (n^6+3n^4+12n^3+8n^2)$
		\end{flushleft}
	\subsection{Recursion}
		\begin{flushleft}
			Sea $f(n) = \sum_{i=1}^{k} a_i f(n-i)$ entonces
			podemos considerar la matriz:
			\[
			\begin{bmatrix}
				f(n) \\
				f(n-1) \\
				\vdots \\
				f(n-k+1)
			\end{bmatrix}
			=
			\begin{bmatrix}
				a_1 & a_2 & \cdots & a_{k-1} & a_{k}\\
				1 & 0 & \cdots & 0 & 0\\
				0 & 1 & \cdots & 0 & 0\\
				\vdots & \vdots & \ddots & \vdots & \vdots\\
				0 & 0 & \cdots & 1 & 0
			\end{bmatrix}
			\begin{bmatrix}
				f(n-1) \\
				f(n-2) \\
				\vdots \\
				f(n-k)
			\end{bmatrix}
			\]
			De aqui podemos calcular $f(n)$ con exponenciación de matrices.
			\[
			\begin{bmatrix}
				f(n) \\
				f(n-1) \\
				\vdots \\
				f(n-k+1)
			\end{bmatrix}
			=
			\begin{bmatrix}
				a_1 & a_2 & \cdots & a_{k-1} & a_{k}\\
				1 & 0 & \cdots & 0 & 0\\
				0 & 1 & \cdots & 0 & 0\\
				\vdots & \vdots & \ddots & \vdots & \vdots\\
				0 & 0 & \cdots & 1 & 0
			\end{bmatrix}
			^{n-k}
			\begin{bmatrix}
				f(k) \\
				f(k-1) \\
				\vdots \\
				f(1)
			\end{bmatrix}
			\]
		\end{flushleft}
	\subsection{Theorems}
		\begin{flushleft}
			\textbf{Koeing's Theorem:} La cardinalidad del emparejamiento maximo de una grafica bipartita
			es igual al minimum vertex cover.\\
			\textbf{Hall's Theorem:} Una grafica bipartita $G$ tiene un emparejamiento que cubre todos los nodos
			de $G$ si y solo si para todo subconjunto $S$ de nodos de $G$, el número de vecinos de $S$ es mayor o igual a $|S|$.\\
			\textbf{Kuratowski's Theorem:} Una grafica es plana si y solo si no contiene un subgrafo homeomorfo a $K_{3,3}$ o $K_5$.\\
		\end{flushleft}

\subsection{Sums}
\[ c^a + c^{a+1} + \dots + c^{b} = \frac{c^{b+1} - c^a}{c-1}, c \neq 1 \]
\begin{align*}
	1 + 2 + 3 + \dots + n &= \frac{n(n+1)}{2} \\
	1^2 + 2^2 + 3^2 + \dots + n^2 &= \frac{n(2n+1)(n+1)}{6} \\
	1^3 + 2^3 + 3^3 + \dots + n^3 &= \frac{n^2(n+1)^2}{4} \\
	1^4 + 2^4 + 3^4 + \dots + n^4 &= \frac{n(n+1)(2n+1)(3n^2 + 3n - 1)}{30} \\
\end{align*}

\subsection{Catalan numbers}
\[ C_n=\frac{1}{n+1}\binom{2n}{n}= \binom{2n}{n}-\binom{2n}{n+1} = \frac{(2n)!}{(n+1)!n!} \]
\[ C_0=1,\ C_{n+1} = \frac{2(2n+1)}{n+2}C_n,\ C_{n+1}=\sum C_iC_{n-i} \]
${C_n = 1, 1, 2, 5, 14, 42, 132, 429, 1430, 4862, 16796, 58786, \dots}$
\begin{itemize}
	\item sub-diagonal monotone paths in an $n\times n$ grid.
	\item strings with $n$ pairs of parenthesis, correctly nested. If prefix is given, number of ways is $\binom{n}{remaining_closed}$-$\binom{n}{remaining_closed+1}$.
	\item binary trees with with $n+1$ leaves (0 or 2 children).
	\item ordered trees with $n+1$ vertices.
	\item ways a convex polygon with $n+2$ sides can be cut into triangles by connecting vertices with straight lines.
	\item permutations of $[n]$ with no 3-term increasing subseq.
\end{itemize}
\
\subsection{Cayley's formula}
Number of labeled trees of n vertices is $n^{n-2}$.
Number of rooted forest of n vertices is $(n+1)^{n-1}$.

\subsection{Geometric series}
Infinite \\
$a+ar+ar^2+ar^3+........+ \sum_{k=0}^{\infty} ar^k $	
\\
Sum=$\frac{a}{1-r}$
\\
Finite \\
$a+ar+ar^2+ar^3+........+ \sum_{k=0}^{n} ar^k $
\\
Sum=$\frac{a(1-r^{n+1})}{1-r}$


\subsection{Estimates For Divisors}

$\sum_{d|n} d = O(n \log \log n)$.

The number of divisors of $n$   is at most around 100 for $n < 5e4$, 500 for $n < 1e7$, 2000 for $n < 1e10$, 200\,000 for $n < 1e19$.

\subsection{Sum of divisors}

$\sum{d|n} =\frac{{p_1}^{\alpha_1+1}-1}{p_1-1}+\frac{{p_2}^{\alpha_2+1}-1}{p_2-1}+....+\frac{{p_n}^{\alpha_n+1}-1}{p_n-1}$

\subsection{Pythagorean Triplets}
The Pythagorean triples are uniquely generated by
\[ a=k\cdot (m^{2}-n^{2}),\ \,b=k\cdot (2mn),\ \,c=k\cdot (m^{2}+n^{2}), \]
with $m > n > 0$, $k > 0$, $m \bot n$, and either $m$ or $n$ even.


\subsection{Derangements}
Permutations of a set such that none of the elements appear in their original position.
\[ \mkern-2mu D(n) = (n-1)(D(n-1)+D(n-2)) = n D(n-1)+(-1)^n = \left\lfloor\frac{n!}{e}\right\rceil \]


\section{Game Theory}
	\subsection{Sprague-Grundy theorem}
	\url{https://codeforces.com/blog/entry/66040}
	Dado un juego con pilas $p_1$, $p_2$, ..., $p_n$ sea $g(p)$ el nimber de la pila $p$, 
	entonces el nimber del juego es $g(p_1) \oplus g(p_2) \oplus ... \oplus g(p_n)$.
	Para calcular el nimber de una pila, se puede usar la fórmula $g(r) = mex(\{g(r_1), g(r_2), ..., g(r_k)\})$
	donde $r_1, r_2, ..., r_k$ son los posibles estados a los que se puede llegar desde $r$ y $g(r)=0$ si $r$ es un estado perdedor.