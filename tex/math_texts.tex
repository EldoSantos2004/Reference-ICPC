\subsection{Identities}
	{
		$C_n = \frac{2(2n-1)}{n+1} C_{n-1}$

		$C_n = \frac{1}{n+1} \binom{2n}{n}$

		$C_n \sim \frac{4^n}{n^{3/2}\sqrt{\pi}}$

		$\sigma(n) = O(\log(\log(n)))$ (number of divisors of $n$)

		$F_{2n+1} = F_{n}^2 + F_{n+1}^2$

		$F_{2n} = F_{n+1}^2 - F_{n-1}^2$

		$\sum_{i=1}^n F_i = F_{n+2}-1$

		$F_{n+i}F_{n+j} - F_nF_{n+i+j} = (-1)^n F_iF_j$

		(Möbius Inv. Formula)
		$\mu(p^k) = [k=0] - [k=1]$
		Let $g(n) = \sum_{d\mid n} f(d)$, then $f(n)=\sum_{d\mid n} g(d) \mu\left(\frac{n}{d}\right)$.

		(Dirichlet Convolution)
		Let $f,g$ be arithmetic functions, then $(f*g)(n) = \sum_{d\mid n} f(d)g\left(\frac{n}{d}\right)$.
		If $f,g$ are multiplicative, then so is $f*g$.

		$n=\sum_{d|n}\phi(d)$

		Lucas' Theorem: $\binom{m}{n} \equiv \prod_{i=0}^k \binom{m_i}{n_i} \pmod{p}$ where $m = \sum_{i=0}^k m_i p^i$ and $n = \sum_{i=0}^k n_i p^i$.
	}
	\subsection{Burnside's Lemma}
		\begin{flushleft}
		Dado un grupo $G$ de permutaciones y un conjunto $X$ de $n$ elementos, 
		el número de órbitas de $X$ bajo la acción de $G$ es igual al promedio
		del número de puntos fijos de las permutaciones en $G$.\\
		Formalmente, el número de órbitas es $\frac{1}{|G|} \sum_{g \in G} f(g)$ donde $f(g)$ es el número de puntos fijos de $g$.\\
		Ejemplo: Dado un collar con $n$ cuentas y $2$ colores, el número de collares
		diferentes que se pueden formar es $\frac{1}{n} \sum_{i=0}^{n} f(i)$ donde $f(i)$ es el número de collares
		que quedan fijos bajo una rotación de $i$ posiciones.\\
		Para contar el número de collares que quedan fijos bajo una rotación de $i$ posiciones, se puede usar
		la fórmula $f(i) = 2^{\gcd(i,n)}$.\\
		Para un collar de $n$ cuentas y $k$ colores, el número de collares diferentes que se pueden formar
		es $\frac{1}{n} \sum_{i=0}^{n} k^{\gcd(i,n)}$\\
		Ejemplo: Dado un cubo con $6$ caras y $k$ colores, el número de cubos diferentes que se pueden formar
		es $\frac{1}{24} \sum_{i=0}^{24} f(i)$ donde $f(i)$ es el número de cubos que quedan fijos bajo una rotación de $i$ posiciones.
		Esta formula es igual a $\frac{1}{24} (n^6+3n^4+12n^3+8n^2)$
		\end{flushleft}
	\subsection{Recursion}
		\begin{flushleft}
			Sea $f(n) = \sum_{i=1}^{k} a_i f(n-i)$ entonces
			podemos considerar la matriz:
			\[
			\begin{bmatrix}
				f(n) \\
				f(n-1) \\
				\vdots \\
				f(n-k+1)
			\end{bmatrix}
			=
			\begin{bmatrix}
				a_1 & a_2 & \cdots & a_{k-1} & a_{k}\\
				1 & 0 & \cdots & 0 & 0\\
				0 & 1 & \cdots & 0 & 0\\
				\vdots & \vdots & \ddots & \vdots & \vdots\\
				0 & 0 & \cdots & 1 & 0
			\end{bmatrix}
			\begin{bmatrix}
				f(n-1) \\
				f(n-2) \\
				\vdots \\
				f(n-k)
			\end{bmatrix}
			\]
			De aqui podemos calcular $f(n)$ con exponenciación de matrices.
			\[
			\begin{bmatrix}
				f(n) \\
				f(n-1) \\
				\vdots \\
				f(n-k+1)
			\end{bmatrix}
			=
			\begin{bmatrix}
				a_1 & a_2 & \cdots & a_{k-1} & a_{k}\\
				1 & 0 & \cdots & 0 & 0\\
				0 & 1 & \cdots & 0 & 0\\
				\vdots & \vdots & \ddots & \vdots & \vdots\\
				0 & 0 & \cdots & 1 & 0
			\end{bmatrix}
			^{n-k}
			\begin{bmatrix}
				f(k) \\
				f(k-1) \\
				\vdots \\
				f(1)
			\end{bmatrix}
			\]
		\end{flushleft}
	\subsection{Theorems}
		\begin{flushleft}
			\textbf{Koeing's Theorem:} La cardinalidad del emparejamiento maximo de una grafica bipartita
			es igual al minimum vertex cover.\\
			\textbf{Hall's Theorem:} Una grafica bipartita $G$ tiene un emparejamiento que cubre todos los nodos
			de $G$ si y solo si para todo subconjunto $S$ de nodos de $G$, el número de vecinos de $S$ es mayor o igual a $|S|$.\\
			\textbf{Kuratowski's Theorem:} Una grafica es plana si y solo si no contiene un subgrafo homeomorfo a $K_{3,3}$ o $K_5$.\\
		\end{flushleft}

\subsection{Sums}
\[ c^a + c^{a+1} + \dots + c^{b} = \frac{c^{b+1} - c^a}{c-1}, c \neq 1 \]
\begin{align*}
	1 + 2 + 3 + \dots + n &= \frac{n(n+1)}{2} \\
	1^2 + 2^2 + 3^2 + \dots + n^2 &= \frac{n(2n+1)(n+1)}{6} \\
	1^3 + 2^3 + 3^3 + \dots + n^3 &= \frac{n^2(n+1)^2}{4} \\
	1^4 + 2^4 + 3^4 + \dots + n^4 &= \frac{n(n+1)(2n+1)(3n^2 + 3n - 1)}{30} \\
\end{align*}

\subsection{Catalan numbers}
\[ C_n=\frac{1}{n+1}\binom{2n}{n}= \binom{2n}{n}-\binom{2n}{n+1} = \frac{(2n)!}{(n+1)!n!} \]
\[ C_0=1,\ C_{n+1} = \frac{2(2n+1)}{n+2}C_n,\ C_{n+1}=\sum C_iC_{n-i} \]
${C_n = 1, 1, 2, 5, 14, 42, 132, 429, 1430, 4862, 16796, 58786, \dots}$
\begin{itemize}
	\item sub-diagonal monotone paths in an $n\times n$ grid.
	\item strings with $n$ pairs of parenthesis, correctly nested. If prefix is given, number of ways is $\binom{n}{remaining_closed}$-$\binom{n}{remaining_closed+1}$.
	\item binary trees with with $n+1$ leaves (0 or 2 children).
	\item ordered trees with $n+1$ vertices.
	\item ways a convex polygon with $n+2$ sides can be cut into triangles by connecting vertices with straight lines.
	\item permutations of $[n]$ with no 3-term increasing subseq.
\end{itemize}
\
\subsection{Cayley's formula}
Number of labeled trees of n vertices is $n^{n-2}$.
Number of rooted forest of n vertices is $(n+1)^{n-1}$.

\subsection{Geometric series}
Infinite \\
$a+ar+ar^2+ar^3+........+ \sum_{k=0}^{\infty} ar^k $	
\\
Sum=$\frac{a}{1-r}$
\\
Finite \\
$a+ar+ar^2+ar^3+........+ \sum_{k=0}^{n} ar^k $
\\
Sum=$\frac{a(1-r^{n+1})}{1-r}$


\subsection{Estimates For Divisors}

$\sum_{d|n} d = O(n \log \log n)$.

The number of divisors of $n$   is at most around 100 for $n < 5e4$, 500 for $n < 1e7$, 2000 for $n < 1e10$, 200\,000 for $n < 1e19$.

\subsection{Sum of divisors}

$\sum{d|n} =\frac{{p_1}^{\alpha_1+1}-1}{p_1-1}+\frac{{p_2}^{\alpha_2+1}-1}{p_2-1}+....+\frac{{p_n}^{\alpha_n+1}-1}{p_n-1}$

\subsection{Pythagorean Triplets}
The Pythagorean triples are uniquely generated by
\[ a=k\cdot (m^{2}-n^{2}),\ \,b=k\cdot (2mn),\ \,c=k\cdot (m^{2}+n^{2}), \]
with $m > n > 0$, $k > 0$, $m \bot n$, and either $m$ or $n$ even.


\subsection{Derangements}
Permutations of a set such that none of the elements appear in their original position.
\[ \mkern-2mu D(n) = (n-1)(D(n-1)+D(n-2)) = n D(n-1)+(-1)^n = \left\lfloor\frac{n!}{e}\right\rceil \]


\section{Game Theory}
	\subsection{Sprague-Grundy theorem}
	\url{https://codeforces.com/blog/entry/66040}
	Dado un juego con pilas $p_1$, $p_2$, ..., $p_n$ sea $g(p)$ el nimber de la pila $p$, 
	entonces el nimber del juego es $g(p_1) \oplus g(p_2) \oplus ... \oplus g(p_n)$.
	Para calcular el nimber de una pila, se puede usar la fórmula $g(r) = mex(\{g(r_1), g(r_2), ..., g(r_k)\})$
	donde $r_1, r_2, ..., r_k$ son los posibles estados a los que se puede llegar desde $r$ y $g(r)=0$ si $r$ es un estado perdedor.


	\section{Fórmulas y notas}
	\subsection{Números de Stirling del primer tipo}
		$\stirlingI{n}{k}$ representa el número de permutaciones de $n$ elementos en exactamente $k$ ciclos disjuntos.
		\begin{align*}
			\stirlingI{0}{0} &= 1 \\
			\stirlingI{0}{n} &= \stirlingI{n}{0} = 0 \quad &, \quad n>0 \\
			\stirlingI{n}{k} &= (n-1)\stirlingI{n-1}{k} + \stirlingI{n-1}{k-1} \quad &, \quad k>0 \\
			\sum_{k=0}^{n} \stirlingI{n}{k} &= n! \\
			\sum_{k=0}^{\infty} \stirlingI{n}{k} x^k &= \prod_{k=0}^{n-1}(x+k)
		\end{align*}
	
	\subsection{Números de Stirling del segundo tipo}
		$\stirlingII{n}{k}$ representa el número de formas de particionar un conjunto de $n$ objetos distinguibles en $k$ subconjuntos no vacíos.
		\begin{align*}
			\stirlingII{0}{0} &= 1 \\
			\stirlingII{0}{n} &= \stirlingII{n}{0} = 0 \quad &, \quad n>0 \\
			\stirlingII{n}{k} &= k\stirlingII{n-1}{k} + \stirlingII{n-1}{k-1} \quad &, \quad k>0 \\
			&= \sum_{j=0}^{k} \dfrac{j^n}{j!} \cdot \dfrac{(-1)^{k-j}}{(k-j)!}
		\end{align*}
	
	\subsection{Números de Euler}
		$\euler{n}{k}$ representa el número de permutaciones de $1$ a $n$ en donde exactamente $k$ números son mayores que el número anterior, es decir, cuántas permutaciones tienen $k$ ``ascensos''.
		\begin{align*}
			\euler{1}{0} &= 1 \\
			\euler{n}{k} &= (n-k)\euler{n-1}{k-1} + (k+1)\euler{n-1}{k} \quad &, \quad n \geq 2 \\
			&= \sum_{j=0}^{k} (-1)^j \binom{n+1}{j} (k+1-j)^n \\
			\sum_{k=0}^{n-1} \euler{n}{k} &= n!
		\end{align*}
	
	\subsection{Números de Catalan}
		\begin{align*}
			C_0 &= 1 \\
			C_n &= \dfrac{1}{n+1}\binom{2n}{n} = \sum_{j=0}^{n-1} C_j C_{n-1-j} \\
			\sum_{n=0}^{\infty} C_n x^n &= \dfrac{1-\sqrt{1-4x}}{2x}
		\end{align*}
	
	\subsection{Números de Bell}
		$B_n$ representa el número de formas de particionar un conjunto de $n$ elementos.
		\begin{align*}
			B_n &= \sum_{k=0}^{n}\stirlingII{n}{k} = \sum_{k=0}^{n-1}\binom{n-1}{k} B_k \\
			\sum_{n=0}^{\infty} \dfrac{B_n}{n!}x^n &= e^{e^x-1}
		\end{align*}
	
	\subsection{Números de Bernoulli}
		\begin{align*}
			{B_0}^+ &= 1 \\
			{B_n}^+ &= 1 - \sum_{k=0}^{n-1}\binom{n}{k}\dfrac{{B_k}^+}{n-k+1} \\
			\sum_{n=0}^{\infty} \dfrac{{B_n}^+ x^n}{n!} &= \dfrac{x}{1-e^{-x}} = \dfrac{1}{\frac{1}{1!}-\frac{x}{2!}+\frac{x^2}{3!}-\frac{x^3}{4!}+\cdots}
		\end{align*}
	
	\subsection{Fórmula de Faulhaber}
		\begin{align*}
			S_p(n) &= \sum_{k=1}^{n}k^p = \dfrac{1}{p+1}\sum_{k=0}^{p} \binom{p+1}{k} {B_k}^+ n^{p+1-k}
		\end{align*}
	
	\subsection{Función Beta}
		\begin{align*}
			B(x,y) &= \dfrac{\Gamma(x)\Gamma(y)}{\Gamma(x+y)} = 2 \int_{0}^{\pi/2} \sin^{2x-1}(\theta) \cos^{2x-1}(\theta) d\theta \\
			&= \int_{0}^{1} t^{x-1} (1-t)^{y-1} dt = \int_{0}^{\infty} \dfrac{t^{x-1}}{(1+t)^{x+y}} dt
		\end{align*}
		
	\subsection{Función zeta de Riemann}
		La siguiente fórmula converge rápido para valores pequeños de $n$ ($n \approx 20$):
		\begin{align*}
			\zeta(s) &\approx \dfrac{1}{d_0 (1 - 2^{1-s})} \sum_{k=1}^{n} \dfrac{(-1)^{k-1} d_k}{k^s} \\
			d_k &= \sum_{j=k}^{n} \dfrac{4^j}{n+j} \binom{n+j}{2j}
		\end{align*}
	
	\subsection{Funciones generadoras}
		\begin{align*}
			\sum_{n=0}^{\infty} \left( \sum_{k=0}^{n}a_k \right) x^n &= \dfrac{1}{1-x}\sum_{n=0}^{\infty} a_n x^n \\
			\sum_{n=0}^{\infty} \binom{n+k-1}{k-1}x^n &= \dfrac{1}{\left(1-x\right)^k} \\
			\sum_{n=0}^{\infty} p_n x^n &= \dfrac{1}{\displaystyle \prod_{k=1}^{\infty}(1-x^k)} = \dfrac{1}{\displaystyle \sum_{n=-\infty}^{\infty} (-1)^n x^{\frac{1}{2}n(3n+1)}} \\
			\sum_{p=0}^{\infty} \dfrac{S_p(n)}{p!} x^p &= \dfrac{e^{x(n+1)}-e^x}{e^x-1} \\
			\sum_{n=0}^{\infty} n^k x^n &= \dfrac{\displaystyle \sum_{i=0}^{k-1} \euler{k}{i} x^{i+1}}{(1-x)^{k+1}} \quad , \quad k \geq 1
		\end{align*}
		Sean $a_1, a_2, \ldots, a_n$ números complejos. Sean $p_m = \displaystyle \sum_{i=1}^{n} a_i^m$ y $s_m$ el $m$-ésimo polinomio elemental simétrico de $a_1, a_2, \ldots, a_n$. Entonces se cumple que $xS'(x) + P(x)S(x) = 0$, donde $P(x)=\displaystyle \sum_{m=1}^{\infty} p_m x^m$ y $S(x)=\displaystyle \prod_{i=1}^{n}(1-a_ix) = \sum_{m=0}^{n}(-1)^m s_m x^m$.
	
	\subsection{Números armónicos}
		\begin{align*}
			H_n &= \sum_{k=1}^{n} \dfrac{1}{k} \approx \ln(n) + \gamma + \dfrac{1}{2n} - \dfrac{1}{12n^2} \\
			\gamma &\approx 0.577215664901532860606512
		\end{align*}
	
	\subsection{Aproximación de Stirling}
		\begin{align*}
			\ln(n!) &\approx n\ln(n) - n + \dfrac{1}{2}\ln(2 \pi n) \\
			\text{\# de dígitos de $n!$} &= 1 + \left\lfloor n\log\left(\dfrac{n}{e}\right) + \dfrac{1}{2}\log(2 \pi n) \right\rfloor \quad \text{($n \geq 30$)}
		\end{align*}
	
	\subsection{Ternas pitagóricas}
		\begin{itemize}
			\item Una terna de enteros positivos $(a,b,c)$ es pitagórica si $a^2+b^2=c^2$. Además es primitiva si $\gcd(a,b,c)=1$.
			\item Generador de ternas primitivas:
			\begin{align*}
				a &= m^2-n^2 \\
				b &= 2mn \\
				c &= m^2+n^2
			\end{align*}
			donde $n \geq 1$, $m>n$, $\gcd(m,n)=1$ y $m,n$ tienen distinta paridad.
			\item Árbol de ternas pitagóricas primitivas: al multiplicar cualquiera de estas matrices:
			\begin{align*}
				\begin{pmatrix}
					1 & -2 & 2 \\
					2 & -1 & 2 \\
					2 & -2 & 3
				\end{pmatrix} \quad , \quad
				\begin{pmatrix}
					-1 & 2 & 2 \\
					-2 & 1 & 2 \\
					-2 & 2 & 3
				\end{pmatrix} \quad , \quad
				\begin{pmatrix}
					1 & 2 & 2 \\
					2 & 1 & 2 \\
					2 & 2 & 3
				\end{pmatrix}
			\end{align*}
			por una terna primitiva $\mathbf{v^T}$, obtenemos otra terna primitiva diferente. En particular, si empezamos con $\mathbf{v}=(3,4,5)$, podremos generar todas las ternas primitivas.
		\end{itemize}

	\subsection{Árbol de Stern–Brocot}
		Todos los racionales positivos se pueden representar como un árbol binario de búsqueda completo infinito con raíz $\frac{1}{1}$.
		\begin{itemize}
			\item Dado un racional $q=[a_0;a_1,a_2,\ldots,a_k]$ donde $a_k \neq 1$, sus hijos serán $[a_0;a_1,a_2,\ldots,a_k+1]$ y $[a_0;a_1,a_2,\ldots,a_k-1,2]$, y su padre será $[a_0;a_1,a_2,\ldots,a_k-1]$.
			\item Para hallar el camino de la raíz $\frac{1}{1}$ a un racional $q$, se usa búsqueda binaria iniciando con $L=\frac{0}{1}$ y $R=\frac{1}{0}$. Para hallar $M$ se supone que $L=\frac{a}{b}$ y $R=\frac{c}{d}$, entonces $M=\frac{a+c}{b+d}$.
		\end{itemize}

	\subsection{Combinatoria}
		\begin{itemize}
			\item Principio de las casillas: al colocar $n$ objetos en $k$ lugares hay al menos $\lceil \frac{n}{k} \rceil$ objetos en un mismo lugar.
			\item Número de funciones: sean $A$ y $B$ conjuntos con $m=\abs{A}$ y $n=\abs{B}$. Sea $f : A \to B$:
			\begin{itemize}
				\item Si $m \leq n$, entonces hay $\displaystyle m!\binom{n}{m}$ funciones inyectivas $f$.
				\item Si $m=n$, entonces hay $n!$ funciones biyectivas $f$.
				\item Si $m \geq n$, entonces hay $n!\stirlingII{m}{n}$ funciones suprayectivas $f$.
			\end{itemize}
			\item Barras y estrellas: ¿cuántas soluciones en los enteros no negativos tiene la ecuación $\displaystyle \sum_{i=1}^{k}x_i = n$? Tiene  $\displaystyle \binom{n+k-1}{k-1}$ soluciones.
			\item ¿Cuántas soluciones en los enteros positivos tiene la ecuación $\displaystyle \sum_{i=1}^{k}x_i = n$? Tiene  $\displaystyle \binom{n-1}{k-1}$ soluciones.
			\item Desordenamientos: $a_0=1$, $a_1=0$, $a_n=(n-1)(a_{n-1}+a_{n-2})=na_{n-1}+(-1)^n$.
			\item Sea $f(x)$ una función. Sea $g_n(x)=x g_{n-1}'(x)$ con $g_0(x)=f(x)$. Entonces $g_n(x)=\sum_{k=0}^{n} \stirlingII{n}{k} x^k f^{(k)}(x)$.
			\item Supongamos que tenemos $m+1$ puntos: $(0, y_0)$, $(1, y_1)$, $\ldots$, $(m, y_m)$. Entonces el polinomio $P(x)$ de grado $m$ que pasa por todos ellos es:
			\begin{align*}
				P(x) &= \left[ \prod_{i=0}^{m}(x-i) \right] (-1)^m \sum_{i=0}^{m} \dfrac{y_i (-1)^i}{(x-i)i!(m-i)!}
			\end{align*}
			\item Sea $a_0, a_1, \ldots$ una recurrencia lineal homogénea de grado $d$ dada por $\displaystyle a_n=\sum_{i=1}^{d} b_i a_{n-i}$ para $n \geq d$ con términos iniciales $a_0, a_1, \ldots, a_{d-1}$. Sean $A(x)$ y $B(x)$ las funciones generadoras de las sucesiones $a_n$ y $b_n$ respectivamente, entonces se cumple que $A(x)=\dfrac{A_0(x)}{1-B(x)}$, donde $\displaystyle A_0(x)=\sum_{i=0}^{d-1} \left[ a_i - \sum_{j=0}^{i-1}a_j b_{i-j} \right] x^i$.
			\item Si queremos obtener otra recurrencia $c_n$ tal que $c_n=a_{kn}$, las raíces del polinomio característico de $c_n$ se obtienen al elevar todas las raíces del polinomio característico de $a_n$ a la $k$-ésima potencia; y sus términos iniciales serán $a_0, a_k, \ldots, a_{k(d-1)}$.
		\end{itemize}

	\subsection{Grafos}
		\begin{itemize}
			\item Sea $d_n$ el número de grafos con $n$ vértices etiquetados: $\displaystyle d_n = 2^{\binom{n}{2}}$.
			\item Sea $c_n$ el número de grafos conexos con $n$ vértices etiquetados. Tenemos la recurrencia: $c_1=1$ y $\displaystyle d_n = \sum_{k=1}^{n} \binom{n-1}{k-1} c_k d_{n-k}$. También se cumple, usando funciones generadoras exponenciales, que $C(x)=1+\ln(D(x))$.
			\item Sea $t_n$ el número de torneos fuertemente conexos en $n$ nodos etiquetados. Tenemos la recurrencia $t_1=1$ y $\displaystyle d_n = \sum_{k=1}^{n} \binom{n}{k} t_k d_{n-k}$. Usando funciones generadoras exponenciales, tenemos que $T(x)=1-\dfrac{1}{D(x)}$.
			\item Número de spanning trees en un grafo completo con $n$ vértices etiquetados: $n^{n-2}$.
			\item Número de bosques etiquetados con $n$ vértices y $k$ componentes conexas: $kn^{n-k-1}$.
			\item Para un grafo no dirigido simple $G$ con $n$ vértices etiquetados de $1$ a $n$, sea $Q=D-A$, donde $D$ es la matriz diagonal de los grados de cada nodo de $G$ y $A$ es la matriz de adyacencia de $G$. Entonces el número de spanning trees de $G$ es igual a cualquier cofactor de $Q$.
			\item Sea $G$ un grafo. Se define al polinomio $P_G(x)$ como el polinomio cromático de $G$, en donde $P_G(k)$ nos dice cuántas $k$-coloraciones de los vértices admite $G$. Ejemplos comunes:
			\begin{itemize}
				\item Grafo completo de $n$ nodos: $P(x)=x(x-1)(x-2) \ldots (x-(n-1))$
				\item Grafo vacío de $n$ nodos: $P(x)=x^n$
				\item Árbol de $n$ nodos: $P(x)=x(x-1)^{n-1}$
				\item Ciclo de $n$ nodos: $P(x)=(x-1)^n + (-1)^n(x-1)$
			\end{itemize}
		\end{itemize}
	
	\subsection{Teoría de números}
		\begin{align*}
			(f * e)(n) &= f(n) \\
			(\varphi * \mathbf{1})(n) &= n \\
			(\mu * \mathbf{1})(n) &= e(n) \\
			\varphi(n^k) &= n^{k-1}\varphi(n) \\
			\sum_{\substack{k=1	\\ \gcd(k,n)=1}}^{n} k &= \dfrac{n \varphi(n)}{2} \quad , \quad n \geq 2 \\
			\sum_{k=1}^{n} \text{lcm}(k,n) &= \dfrac{n}{2} + \dfrac{n}{2}\sum_{d | n} d\varphi(d) = \dfrac{n}{2} + \dfrac{n}{2} \prod_{p^a | n} \dfrac{p^{2a+1}+1}{p+1} \\
			\sum_{k=1}^{n} \gcd(k,n) &= \sum_{d | n} d\varphi\left(\dfrac{n}{d}\right) = \prod_{p^a | n} p^{a-1}(1+(a+1)(p-1))
		\end{align*}
	
		\begin{itemize}
			\item Lifting the exponent: sea $p$ un primo, $x,y$ enteros y $n$ un entero positivo tal que $p \mid x-y$ pero $p \nmid x$ ni $p \nmid y$. Entonces:
			\begin{itemize}
				\item Si $p$ es impar: $v_p(x^n-y^n) = v_p(x-y) + v_p(n)$
				\item Si $p=2$ y $n$ es par: $v_p(x^n-y^n) = v_p(x-y) + v_p(n) + v_p(x+y) - 1$
			\end{itemize}
			donde $v_p(n)$ es el exponente de $p$ en la factorización en primos de $n$.
			\item Suma de dos cuadrados: sea $\chi_4(n)$ una función multiplicativa igual a 1 si $n \equiv 1 \mod 4$, $-1$ si $n \equiv 3 \mod 4$ y cero en otro caso. Entonces, el número de soluciones enteras $(a,b)$ de la ecuación $a^2+b^2=n$ es $4(\chi_4 * 1)(n) = 4 \displaystyle \sum_{d | n} \chi_4(d)$.
			\item Teorema de Lucas:
			\begin{align*}
				\binom{m}{n} &\equiv \prod_{i=0}^{k} \binom{m_i}{k_i} \pmod{p} \\
				m = \sum_{i=0}^{k} m_i p^i \quad &, \quad n = \sum_{i=0}^{k} n_i p^i \\
				0 \leq m_i &, n_i < p
			\end{align*}
			
			\item Sean $a,b,c \in \mathbb{Z}$ con $a \neq 0$ y $b \neq 0$. La ecuación $ax+by=c$ tiene como soluciones:
			\begin{align*}
				x &= \dfrac{x_0 c - bk}{d} \\
				y &= \dfrac{y_0 c + ak}{d} 
			\end{align*}
			para toda $k \in \mathbb{Z}$ si y solo si $d | c$, donde $ax_0+by_0=\gcd(a,b)=d$ (Euclides extendido). Si $a$ y $b$ tienen el mismo signo, hay exactamente $\max\left( \left\lfloor\dfrac{x_0 c}{\abs{b}}\right\rfloor + \left\lfloor\dfrac{y_0 c}{\abs{a}}\right\rfloor + 1, 0 \right)$ soluciones no negativas. Si tienen el signo distinto, hay infinitas soluciones no negativas.
			
			\item Dada una función aritmética $f$ con $f(1) \neq 0$, existe otra función aritmética $g$ tal que $(f*g)(n)=e(n)$, dada por:
			\begin{align*}
				g(1) &= \dfrac{1}{f(1)} \\
				g(n) &= -\dfrac{1}{f(1)} \sum_{d | n, d<n} f\left(\dfrac{n}{d}\right)g(d) \quad , \quad n > 1
			\end{align*}
			
			\item Sean $\displaystyle h(n) = \sum_{k=1}^{n} f\left(\left\lfloor \dfrac{n}{k} \right\rfloor\right) g(k)$, $\displaystyle G(n)=\sum_{k=1}^{n}g(k)$ y $m=\left\lfloor \sqrt{n} \right\rfloor$, entonces:
			\begin{align*}
				h(n) &= \sum_{k=1}^{\lfloor n/m \rfloor}f\left(\left\lfloor \dfrac{n}{k} \right\rfloor\right) g(k) + \sum_{k=1}^{m-1} \left( G\left(\left\lfloor \dfrac{n}{k} \right\rfloor\right) - G\left(\left\lfloor \dfrac{n}{k+1} \right\rfloor\right) \right)f(k)
			\end{align*}
			
			\item Sean $\displaystyle F(n)=\sum_{k=1}^{n}f(k)$, $\displaystyle G(n)=\sum_{k=1}^{n}g(k)$, $\displaystyle h(n)=(f * g)(n)=\sum_{d | n}f(d)g\left(\dfrac{n}{d}\right)$ y $\displaystyle H(n)=\sum_{k=1}^{n}h(k)$, entonces:
			\begin{align*}
				H(n) &= \sum_{k=1}^{n}f(k)G\left(\left\lfloor \dfrac{n}{k} \right\rfloor\right)
			\end{align*}
			
			\item Sean $\displaystyle \Phi_p(n) = \sum_{k=1}^{n}k^p\varphi(k)$ y $\displaystyle M_p(n) = \sum_{k=1}^{n}k^p\mu(k)$. Aplicando lo anterior, podemos calcular $\Phi_p(n)$ y $M_p(n)$ con complejidad $O(n^{2/3})$ si precalculamos con fuerza bruta los primeros $\lfloor n^{2/3} \rfloor$ valores, y para los demás, usamos las siguientes recurrencias (DP con \texttt{map}):
			{\small
			\begin{align*}
				\Phi_p(n) &= S_{p+1}(n) - \sum_{k=2}^{\lfloor n/m \rfloor} k^p \Phi_p\left(\left\lfloor \dfrac{n}{k}  \right\rfloor\right) - \sum_{k=1}^{m-1} \left( S_p\left(\left\lfloor \dfrac{n}{k} \right\rfloor\right) - S_p\left(\left\lfloor \dfrac{n}{k+1} \right\rfloor\right) \right)\Phi_p(k) \\
				M_p(n) &= 1 - \sum_{k=2}^{\lfloor n/m \rfloor} k^p M_p\left(\left\lfloor \dfrac{n}{k}  \right\rfloor\right) - \sum_{k=1}^{m-1} \left( S_p\left(\left\lfloor \dfrac{n}{k} \right\rfloor\right) - S_p\left(\left\lfloor \dfrac{n}{k+1} \right\rfloor\right) \right)M_p(k)
			\end{align*}
			}
			
			\item En general, si queremos hallar $F(n)$ y existe una función mágica $g(n)$ tal que $G(n)$ y $H(n)$ se puedan calcular en $O(1)$, entonces:
			{\small
			\begin{align*}
				F(n) &= \dfrac{1}{g(1)} \left[ H(n) - \sum_{k=2}^{\lfloor n/m \rfloor} g(k)F\left(\left\lfloor \dfrac{n}{k} \right\rfloor\right) - \sum_{k=1}^{m-1} \left( G\left(\left\lfloor \dfrac{n}{k} \right\rfloor\right) - G\left(\left\lfloor \dfrac{n}{k+1} \right\rfloor\right) \right)F(k) \right]
			\end{align*}
			}
		\end{itemize}
		
	\subsection{Primos}
		$10^2+1$, $10^3+9$, $10^4+7$, $10^5+3$, $10^6+3$, $10^7+19$, $10^8+7$, $10^9+7$, $10^{10}+19$, $10^{11}+3$, $10^{12}+39$, $10^{13}+37$, $10^{14}+31$, $10^{15}+37$, $10^{16}+61$, $10^{17}+3$, $10^{18}+3$.
		
		$10^2-3$, $10^3-3$, $10^4-27$, $10^5-9$, $10^6-17$, $10^7-9$, $10^8-11$, $10^9-63$, $10^{10}-33$, $10^{11}-23$, $10^{12}-11$, $10^{13}-29$, $10^{14}-27$, $10^{15}-11$, $10^{16}-63$, $10^{17}-3$, $10^{18}-11$.
	
	\subsection{Números primos de Mersenne}
		Números primos de la forma $M_p=2^p-1$ con $p$ primo. Todos los números perfectos pares son de la forma $2^{p-1}M_p$ y viceversa.
	
		Los primeros 47 valores de $p$ son: 2, 3, 5, 7, 13, 17, 19, 31, 61, 89, 107, 127, 521, 607, 1279, 2203, 2281, 3217, 4253, 4423, 9689, 9941, 11213, 19937, 21701, 23209, 44497, 86243, 110503, 132049, 216091, 756839, 859433, 1257787, 1398269, 2976221, 3021377, 6972593, 13466917, 20996011, 24036583, 25964951, 30402457, 32582657, 37156667, 42643801, 43112609.
	
	\subsection{Números primos de Fermat}
		Números primos de la forma $F_p=2^{2^p}+1$, solo se conocen cinco: 3, 5, 17, 257, 65537. Un polígono de $n$ lados es construible si y solo si $n$ es el producto de algunas potencias de dos y distintos primos de Fermat.
